\documentclass[11pt,a4paper]{report}
\usepackage[a4paper, top=3cm, bottom=3cm]{geometry}
\usepackage{setspace}
\usepackage{fancyhdr}
\setlength{\headheight}{15pt}
\usepackage[brazilian]{babel}
\usepackage[utf8]{inputenc}
\usepackage{listings}
\usepackage{multicol}
\usepackage{courier}
\usepackage{graphicx}
\usepackage{hyperref}
\usepackage{graphicx}
\usepackage[shortlabels]{enumitem}
\hypersetup{
    colorlinks,
    citecolor=black,
    filecolor=black,
    linkcolor=black,
    urlcolor=black
}

\lstset{
	basicstyle=\scriptsize\ttfamily,
	numbers=left,
	numberstyle=\tiny,
	%stepnumber=2,
	numbersep=5pt,
	tabsize=4,
	extendedchars=true,
	breaklines=true,
	keywordstyle=\color{red},
	frame=b,
	stringstyle=\color{white}\ttfamily,
	showspaces=false,
	showtabs=false,
	xleftmargin=17pt,
	framexleftmargin=17pt,
	framexrightmargin=5pt,
	framexbottommargin=4pt,
	showstringspaces=false
}

\begin{document}

\pagestyle{empty}
\title{\textbf{Processador TRM - Manual de referência}}
\author{Sérgio Johann Filho}

\maketitle

\thispagestyle{empty}
\thanks{Essa mensagem de agredecimento foi intencionalmente deixada em branco.}
\newpage

\tableofcontents

\pagestyle{fancy}
\fancyhf{}
\lhead[]{\thepage}
\rhead[\thepage]{}
\pagenumbering{arabic}
%\onehalfspacing

\newpage
\thispagestyle{empty}

\chapter{A função de um computador}

Este capítulo apresenta a função de um sistema computacional, apresentando
conceitos básicos que serão úteis para o entendimento do restante desse manual.

\section{Um sistema computacional típico}

Um sistema computacional típico consiste dos seguintes elementos principais:

\begin{itemize}
\item Uma unidade central de processamento (CPU)
\item Memória para armazenamento do programa
\item Portas de entrada e saída
\end{itemize}

A memória é usada para armazenar \textit{instruções}, que são comandos
que definem as atividades a serem executadas pela CPU, assim como
\textit{dados}, que são informações processadas pelas instruções
executadas. Um agrupamento de instruções e dados logicamente
relacionados e armazenados na memória são chamados de \textit{programa}.
A CPU lê cada instrução da memória em uma sequência determinada e utiliza
essa sequência para realizar o processamento, executando as instruções.
Se as instruções lidas e executadas pela CPU forem coerentes e lógicas,
o processamento do programa irá produzir resultado válido.

A CPU pode acessar qualquer informação na memória, no entanto a memória
nem sempre possui tamanho o suficiente para armazenar toda a informação
necessária para uma aplicação específica. Isso pode ser resolvido por
meio de uma ou mais \textit{portas}. A CPU pode endereçar essas portas
e acessar dados por meio destas, de maneira que torna-se possível enviar
e receber dados de dispositivos periféricos. As portas podem ser de
entrada ou saída. Uma porta de entrada pode ser usada para leitura de
informação externa (de um dispositivo periférico como um teclado, por
exemplo) e uma porta de saída pode ser usada para escrita de informação
para o exterior (para um periférico como um monitor, por exemplo).

Em suma, a CPU é responsável por unificar todas as partes de um sistema
computacional. Ela controla as funções realizadas por outros componentes,
sequenciando todos os eventos. A CPU deve poder buscar instruções,
decodificar estas e as executar. Ela também pode referenciar a memória
para acessos (leitura e escrita de variáveis, por exemplo), assim como
portas para comunicação com dispositivos periféricos.

\section{A arquitetura de uma CPU}

Uma unidade central de processamento tipicamente consiste das seguintes
unidades funcionais interconectadas:

\begin{itemize}
\item Registradores
\item Unidade lógica e aritmética (ALU)
\item Unidade de controle
\end{itemize}

Cada computador possui um \textit{tamanho de palavra} que é uma
característica da máquina em questão. A largura da palavra de um
computador é determinada pelo tamanho dos elementos de armazenamento
interno (registradores) e caminhos de interconexão (barramentos).
Por exemplo, um computador que pode armazenar e transferir 16 bits de
informação por vez possui um tamanho de palavra de 16 bits, e é 
referenciado como um processador de 16 bits. Dados e instruções são
armazenados na memória com um formato de múltiplos de 8 bits (um byte),
ou seja, 8 bits, 16 bits, 24 bits, etc. Um computador cujo tamanho da
palavra seja um múltiplo de 8 bits geralmente processa de maneira mais
eficiente dados com esse tamanho. Em algumas implementações, uma
instrução possui exatamente o tamanho da palavra da arquitetura, em
outras, uma instrução é composta por uma sequência de bytes ou múltiplas
palavras.

O formato das instruções depende da maneira como a arquitetura é definida,
o que inclui o número de instruções, tamanho da palavra, número de
\textit{registradores}, modos de endereçamento, estratégia de
sequenciamento para execução de instruções, entre outros detalhes. Muitos
desses detalhes são expostos no \textit{formato de instrução}, composto
por diversos campos de bits que podem ser decodificados pelo processador
durante o processo de execução de cada instrução.

Registradores são unidades para o armazenamento de informação de forma
temporária. Alguns registradores, como o contador de programa (PC) e
outros possuem um papel especial. Outros registradores, possuem um
propósito mais genérico e normalmente podem ser usados de forma flexível
pelo programador.

\subsection{Registradores de propósito geral (GPRs)}

Os registradores de propósito geral armazenam operandos a serem manipulados
pela \textit{unidade lógica e aritmética} (ALU), bem como o resultado
do processamento de uma instrução. Por exemplo, uma instrução pode
solicitar uma operação sobre de dois valores à ALU, e armazenar o
resultado. De forma geral, os registradores de propósito geral (GPRs)
ão implementados em um vetor (banco) de registradores. Em uma operação
típica, são selecionados dois registradores com os operandos (fontes) a
serem processados e um registrador (destino) para o armazenamento do
resultado.

De maneira geral, são incluídos em uma CPU registradores que podem ser
usados para o armazenamento de operandos e resultados intermediários
do processamento. A disponibilidade de registradores de propósito geral
elimina a necessidade de acessos adicionais à memória, o que aumenta
a velocidade do processamento e a sua eficiência, uma vez que acessos
à memória tendem a ser mais lentos que acessos à registradores\footnote{
Acessos adicionais à memória independem dos acessos realizados para a
execução das instruções. Além disso, podem existir complicações
adicionais relacionadas ao desempenho do sistema de memória.}.

\subsection{Contador de programa (PC)}

As instruções que compõem um programa são armazenadas na memória e a CPU
referencia o conteúdo da memória para acessar as mesmas durante a execução
do programa. Para acessar uma instrução é necessário que a CPU saiba o
endereço da próxima instrução a ser trazida da memória.

No processador é mantido uma referência ao endereço da próxima instrução,
sendo tal referência armazenada em um registrador chamado \textit{contador
de programa} (PC). Esse registrador é atualizado antes da busca de cada
instrução.

As instruções de um programa são armazenadas sequencialmente na memória,
de tal forma que endereços menores armazenam as primeiras instruções de
um programa, já endereços maiores instruções posteriores. Essa regra pode
ser modificada caso sejam utilizadas instruções de desvio (ou saltos) no
programa. Uma instrução de desvio (transferência de controle) contém uma
referência ao endereço da instrução a ser executada no caso do desvio ser
tomado. A próxima instrução pode ser armazenada em qualquer lugar da
memória. Durante a execução de uma instrução de desvio, o processador
substitui o conteúdo do registrador PC com o endereço da próxima
instrução, seja ela o alvo do desvio (caso tomado) ou a instrução
seguinte (caso não tomado).

Um tipo especial de transferência de fluxo ocorre durante a execução
de subprogramas. Nesse caso, o processador deve lembrar do endereço que
deve retornar ao término do subprograma e pode assim continuar a execução
do programa principal.

O processador possui uma maneira especial de executar subprogramas. Antes
de realizar o desvio para o subprograma, é necessário calcular o endereço
de retorno e armazenar o mesmo em um registrador ou na memória, em um
espaço conhecido como \textit{pilha}. Após, o endereço do subprograma
é colocado no PC por meio de uma instrução de desvio incondicional (salto)
e a primeira instrução do subprograma pode ser buscada na memória. A
última instrução em um subprograma é uma instrução de retorno. Essa
instrução também é um desvio incondicional, que restaura o valor do PC
para o endereço salvo anteriormente em um registrador ou na pilha antes
de retomar a execução do programa principal. Registradores ou a pilha
também são utilizados para passagem de argumentos à subprogramas ou para
o retorno do resultado do processamento.

\subsection{Registrador de endereço}

A CPU pode utilizar registradores para o armazenamento de enderereços
de memória a serem acessados para a transferência de dados. Se um
registrador de endereço for programável\footnote{Por meio de uma ou
mais instruções o programador pode modificar o conteúdo do registrador.},
o programa pode criar um endereço no registrador antes de executar uma
operação de referência à memória (uma instrução que lê ou escreve dados
na memória). Essas operações implementam o acesso à variáveis e constantes.

Registradores de endereço podem ser implementados no próprio banco de
registradores, ou seja, podem ser utilizados registradores de propósito
geral para essa finalidade.

\subsection{Unidade lógica e aritmética}

A unidade lógica e aritmética (ALU) é uma parte da CPU responsável pela
operação aritmética e lógica sobre dados. A ALU possui um circuito
\textit{somador}, responsável por combinar o conteúdo de dois registradores
de acordo com as regras de aritmética binária. Isso permite que o
processador realize operações aritméticas sobre dados obtidos da memória
ou portas de entrada.

Utilizando apenas um somador, um bom programador pode escrever rotinas
capazes de realizar a subtração, multiplicação e divisão, o que define
capacidade aritmética completa ao computador. Na prática, a maioria das
ALUs implementam também outras funções, como operações booleanas lógicas
e operações de deslocamento.

Além disso, a ALU possui sinais (\textit{flags}) que especificam
determinadas condições durante a execução de operações lógicas e
aritméticas\footnote{Nem sempre os sinais de condição (\textit{flags} ou
qualificadores de estado) são apresentados ao programador, ou seja, podem 
ser apenas utilizados internamente durante o processamento.}. É possível
dessa forma realizar a comparação de valores e desvios condicionais,
dependendo dos valores de tais sinais após o cálculo do resultado da
última operação.

\subsection{Decodificação de instruções e unidade de controle}

O processador busca uma instrução na memória de programa fornecendo à
mesma um endereço armazenado no \textit{contador de programa} (PC).
A memória retorna a instrução para o processador, que a armazena no
\textit{registrador de instrução} e passa a processar essa por meio
de uma sequência de passos.

Após a busca, a instrução armazenada é decodificada, sendo essa a
principal responsabilidade da \textit{unidade de controle}. De acordo
com o tipo de instrução, a unidade de controle será responsável por
gerar sinais para a ativação de partes do processador, responsáveis
pela execução de cada instrução.

A instrução armazenada no registrador de instrução pode ser decodificada
e utilizada para ativar as diferentes partes internas do \textit{caminho
de dados}. Cada parte interna está associada com a execução de uma
instrução particular, e a habilitação ou não de cada parte, assim como
o sequenciamento das operações é responsável pelo processamento aritmético,
lógico ou de transferência, que define o resultado da operação de uma
instrução.

\section{Operação de um processador}

Todas as operações de um processador ocorrem de forma cíclica, ou seja,
o processador busca uma instrução, realiza sua decodificação, a executa,
produzindo o resultado da operação e volta a buscar a próxima instrução,
repetindo o ciclo. A execução de uma instrução, composta por essas etapas
é chamada de \textit{ciclo de instrução}.

\subsection{Busca e execução de instrução}

A primeira operação no ciclo de instrução é chamado de \textit{busca de
instrução}, que compreende em trazer a próxima instrução a ser executada
da memória. A CPU sinaliza uma operação de leitura e o conteúdo do 
contador de programa (PC) é enviado à memória, que responde com a
instrução endereçada ou parte dessa\footnote{Algumas instruções precisam
de múltiplos acessos à memória para serem buscadas.}. A primeira parte
de uma instrução é colocada no registrador de instrução, e caso a
instrução consista de mais partes, ciclos adicionais de busca são
realizados. No momento em que toda a instrução tiver sido buscada, o
contador de programa (PC) é incrementado e a instrução é decodificada. A
operação especificada na instrução será executada nos estágios seguintes,
sendo que a instrução pode realizar acessos adicionais à memória (no 
caso de instruções que referenciam dados, ou portas de entrada e saída).
Em instruções sem acessos adicionais, como operações aritméticas ou 
lógicas que endereçam apenas registradores, o próximo passo é a execução,
que computará o resultado.

\subsection{Acessos adicionais à memória}

Uma operação de busca é um tipo especial de operação de leitura realizada
na memória, onde a instrução é trazida para o processador para ser
executada. Uma instrução pode solicitar acessos adicionais à memória
(para acesso à dados). Nesse caso, a CPU gera novamente uma operação
de leitura ou escrita, enviando o endereço solicitado pela instrução
à memória. Em uma operação de leitura, o dado é transferido para um
registrador de propósito geral e em uma operação de escrita, o dado
armazenado em um registrador é escrito na memória.


\chapter{A arquitetura TRM}

O processador TRM (Tiny RISC Machine) possui uma arquitetura planejada
de acordo com a filosofia RISC. Essa arquitetura foi definida com o
objetivo de oferecer ao programador um conjunto de instruções básico, de
maneira que uma quantidade reduzida de recursos de hardware seja
necessária para implementar seu conjunto de instruções\footnote{O conjunto
de instruções foi definido com o intuito de minimizar a complexidade da
arquitetura e ao mesmo tempo permitir a síntese de operações mais
complexas por meio de poucas operações simples.}. Entre as decisões de
projeto, foi utilizado um banco de registradores com propósito geral,
não há qualificadores de estado (flags) para operações, não foram
definidas unidades para multiplicação e divisão e há apenas um conjunto
limitado de modos de endereçamento.

Um pequeno conjunto de instruções foi definido na arquitetura TRM (25
instruções básicas). Apesar do pequeno número de instruções, estas são 
poderosas o suficiente para realizar todas as operações de máquinas com
um maior número de instruções. Para isso, em algumas situações uma
instrução pode ser utilizada de forma pouco ortodoxa ou uma combinação
de instruções pode ser utilizada para implementação de um comportamento
mais elaborado. As instruções são separadas nas seguintes classes:

\begin{enumerate}
\item \textit{Lógica e aritmética} (and, or, xor, add, sub)
\item \textit{Comparação} (tlt, tge, tbl, tae, teq, tne)
\item \textit{Deslocamento} (lsl, lsr, asr)
\item \textit{Transferência de dados} (ldw, ldb, lbu, stw, stb)
\item \textit{Transferência de controle} (blt, bge, bbl, bae, beq, bne)
\end{enumerate}

Seguindo a filosofia RISC ao extremo, as instruções são definidas em uma
representação binária que utiliza apenas um formato de
instrução\footnote{Uma instrução pode ou não ser seguida de uma palavra
para o armazenamento de um valor imediato, que atua como uma constante.}.
Dessa forma, a lógica necessária para a decodificação de instruções é reduzida
significativamente, comparado ao que seria necessário para decodificação
de instruções com múltiplos formatos (como nas arquiteturas ARM e RISC-V
por exemplo) ou instruções com tamanho variável (como na arquitetura x64
por exemplo).

\section{Registradores}

Assim como outros processadores RISC, o TRM utiliza operações de
transferência de dados (\textit{load / store}) para acesso à variáveis.
Para que operações de lógica e aritmética, comparação, deslocamento, entre
outras possam ser execututadas, é necessário que os operandos\footnote{
Operandos são valores utilizados para entrada para uma computação.}
sejam  primeiramente trazidos da memória em um ou mais registradores de
propósito geral (GPRs), ou que sejam utilizadas constantes no próprio
fluxo de instruções.

São definidos 16 registradores (\textit{r0 - r15}) e estes podem ser
utilizados para qualquer finalidade\footnote{Exceto \textit{r0}, que
possui o seu valor constante zero.}, sendo apenas recomendado seu uso em
função das convenções apresentadas abaixo. Ao utilizar diferentes
registradores em um programa, um programador tipicamente referencia os
mesmos por seu nome, onde é indicada a sua função. A representação dos
registradores pelo seu nome facilita a leitura do código pelo programador.

\begin{table}[htb!]
\centering
\begin{tabular}{|p{2.5cm}|p{2.5cm}|p{6.0cm}|}
\hline
\bf{Registrador} & \bf{Nome} & \bf{Função} 		\\ \hline \hline
\texttt{r0}	& \texttt{zr}		& Constante zero	\\ \hline
\texttt{r1}	& \texttt{a0}		& Argumentos	 	\\ \hline
\texttt{r2}	& \texttt{a1}		& 	 		\\ \hline
\texttt{r3}	& \texttt{a2}		& 	 		\\ \hline
\texttt{r4}	& \texttt{a3}		& 	 		\\ \hline
\texttt{r5}	& \texttt{v0}		& Variáveis locais	\\ \hline
\texttt{r6}	& \texttt{v1}		& 		 	\\ \hline
\texttt{r7}	& \texttt{v2}		& 		 	\\ \hline
\texttt{r8}	& \texttt{v3}		& 			\\ \hline
\texttt{r9}	& \texttt{v4}		& 		 	\\ \hline
\texttt{r10}	& \texttt{v5}		& 	 		\\ \hline
\texttt{r11}	& \texttt{v6}		& 	 		\\ \hline
\texttt{r12}	& \texttt{v7}		& 	 		\\ \hline
\texttt{r13}	& \texttt{v8/fp}	& Ponteiro de moldura	\\ \hline
\texttt{r14}	& \texttt{sp}		& Ponteiro de pilha 	\\ \hline
\texttt{r15}	& \texttt{lr}		& Endereço de retorno 	\\ \hline
\end{tabular}
\end{table}

Além dos 16 registradores de propósito geral (GPRs), é definido na
arquitetura um registrador com a finalidade de contador de programa (PC).
Esse registrador aponta para a instrução corrente do programa, e não pode
ser modificado diretamente pelo programador. A cada instrução decodificada,
o PC avança para a próxima posição. Instruções de transferência
de fluxo (desvios) podem fazer com que o PC seja atualizado com o
destino do desvio, caso este seja tomado. As instruções possuem um
tamanho de 16 bits (ou 32 bits para instruções com valores imediatos),
portanto o contador de programa é incrementado com esse tamanho\footnote{
Instruções que possuem um operando imadiato armazenam esse
operando como uma constante na próxima posição de memória após a
instrução, dessa forma essas instruções causam um avanço de duas
posições no PC.}.

\section{Formato de instruções}

Existe apenas um formato de instrução na arquitetura TRM, sendo que uma
instrução pode conter um campo imediato opcional, armazenado na palavra
seguinte no fluxo de instruções. No formato, são definidos campos que
descrevem a instrução, a presença ou não de um campo imediato e dois
registradores, utilizados como fonte ou destino das operações.

\begin{table}[htb!]
\centering
\begin{tabular}{|p{2cm}|p{2cm}|p{2cm}|p{2cm}|p{2cm}|}
\hline
$I_{<15:13>}$ & $I_{<12:9>}$ & $I_{<8>}$ & $I_{<7:4>}$ & $I_{<3:0>}$  \\ \hline
\textit{Opcode} & \textit{Opcode 2} & \textit{Imm} & \textit{Reg A} & \textit{Reg B} \\ \hline
\texttt{x x x} & \texttt{x x x x} & \texttt{i} & \texttt{r r r r} & \texttt{r r r r} \\ \hline
\end{tabular}
\end{table}

Os campos \textit{Opcode} e \textit{Opcode 2} definem o tipo de operação
e a operação específica respectivamente. O campo \textit{Imm} define se
a instrução possui ou não um campo imediato. Os campos \textit{Reg A} e 
\textit{Reg B} definem os registradores a serem utilizados na instrução.

\subsection{Instruções regulares}

Em instruções regulares (sem um campo imediato), são referenciados dois
registradores, sendo estes utilizados como fonte ou destino da operação,
dependendo da instrução específica. Exemplos:

\begin{table}[htb!]
\centering
\begin{tabular}{|p{5.0cm}|p{8.0cm}|}
\hline
\bf{Operação} & \bf{Significado} \\ \hline \hline
\texttt{add r1,r2} & \verb|r1 = r1 + r2| \\ \hline
\texttt{xor r1,r1} & \verb|r1 = 0| \\ \hline
\texttt{ldw r1,r2} & \verb|r1 = MEM[r2]| \\ \hline
\texttt{blt r1,r2} & \verb|pc = r1 < 0 ? r2 : pc + 2| \\ \hline
\end{tabular}
\end{table}

\subsection{Instruções com campo imediato}

Em instruções com um campo imediato, são referenciados dois registradores,
sendo estes utilizados como fonte ou destino da operação, dependendo
da instrução específica. Além disso, um campo imediato (constante)
associado à instrução é usado como fonte. Exemplos:

\begin{table}[htb!]
\centering
\begin{tabular}{|p{5.0cm}|p{8.0cm}|}
\hline
\bf{Operação} & \bf{Significado} \\ \hline \hline
\texttt{add r1,r1,1} & \verb|r1 = r1 + 1| \\ \hline
\texttt{lsl r1,r2,3} & \verb|r1 = r2 << 3| \\ \hline
\texttt{bge r1,r2,12340} & \verb|pc = r1 >= r2 ? 12340 : pc + 2| \\ \hline
\texttt{stw r1,r2,12} & \verb|MEM[r2 + 12] = r1| \\ \hline
\end{tabular}
\end{table}


\section{O conjunto de instruções}

O conjunto de instruções definido na arquitetura é apresentado a seguir.
Diversos códigos de operação são reservados para extensões futuras. O
comportamento do conjunto de instruções pode ser sumarizado de acordo
com as observações:

\begin{itemize}
\item Em operações lógicas e aritméticas, comparações e operações de deslocamento
que possuem o formato de instrução regular (sem campo imediato),
\textit{Reg A} é utilizado como operando \textit{fonte 1} e como destino
para o resultado da operação e \textit{Reg B} é utilizado como operando
\textit{fonte 2}. Nas operações com formato imediato, \textit{Reg A} é
utilizado como \textit{destino} para o resultado, \textit{Reg B} é
utilizado como operando \textit{fonte 1} e a palavra seguinte à instrução
contendo o valor imaditato é utilizada como operando \textit{fonte 2}.

\item Em operações de transferência de dados que possuem o formato de instrução
regular, \textit{Reg A} é utilizado como operando \textit{fonte 1}
(instruções de armazenamento) e como \textit{destino} para o resultado
da operação (instruções de carga) e \textit{Reg B} é utilizado como
operando \textit{fonte 2}, contendo o endereço da memória. Nas operações
com formato imediato, Os registradores possuem o mesmo papel, no entanto
é somado um valor imediato ao valor armazenado no operando \textit{fonte 2}
(\textit{Reg B}), sendo realizado o cálculo do endereço \textit{base
+ deslocamento}.

\item Em operações de transferência de controle que possuem o formato de
instrução regular (sem campo imediato), \textit{Reg A} é utilizado como
operando \textit{fonte 1}, \textit{zero} é utilizado como \textit{fonte 2}
e \textit{Reg B} é utilizado como endereço da memória. Nas operações com
formato imediato, \textit{Reg A} é utilizado \textit{fonte 1},
\textit{Reg B} é utilizado como operando \textit{fonte 2} e a palavra
seguinte à instrução contendo o valor imaditato é utilizada como
endereço da memória.
\end{itemize}

A tabela abaixo apresenta um quadro das instruções implementadas na
arquitetura\footnote{A instrução \textit{hlt} não é uma instrução real
da arquitetura. Ela é uma pseudo instrução utilizada para parar a
execução de programas no simulador.}. São apresentados os nomes e códigos
de operação, a descrição e o resultado da instrução nos formatos regular
e imediato.

\begin{table}[htb!]
\centering
\scriptsize
\begin{tabular}{|p{1.5cm}|p{1.5cm}|p{3.5cm}|p{3.0cm}|p{3.0cm}|}
\hline
\bf{Instrução} 	& \bf{Opcode} & \bf{Descrição} 			& \bf{Regular} 			& \bf{Imediato} \\ \hline \hline
and	&\verb|000 0000|	&logical and			&\verb|ra = ra & rb|			&\verb|ra = rb & imm|		\\ \hline
or	&\verb|000 0010|	&logical or			&\verb$ra = ra | rb$			&\verb$ra = rb | imm$		\\ \hline
xor	&\verb|000 0011|	&logical exclusive or		&\verb|ra = ra ^ rb|			&\verb|ra = rb ^ imm|		\\ \hline
add	&\verb|000 0100|	&arithmetic add			&\verb|ra = ra + rb|			&\verb|ra = rb + imm|		\\ \hline
sub	&\verb|000 0101|	&arithmetic subtract		&\verb|ra = ra - rb|			&\verb|ra = rb - imm|		\\ \hline
tlt	&\verb|101 0000|	&test if less than		&\verb|ra = ra < rb ?|\newline\verb|1 : 0|		&\verb|ra = rb < imm ?|\newline\verb|1 : 0|	\\ \hline
tge	&\verb|101 0001|	&test if greater or equal	&\verb|ra = ra >= rb ?|\newline\verb|1 : 0|		&\verb|ra = rb >= imm ?|\newline\verb|1 : 0|	\\ \hline
tbl	&\verb|101 0100|	&test if below			&\verb|ra = ra < rb ?|\newline\verb|1 : 0|		&\verb|ra = rb < imm ?|\newline\verb|1 : 0|	\\ \hline
tae	&\verb|101 0101|	&test if above or equal		&\verb|ra = ra >= rb ?|\newline\verb|1 : 0|		&\verb|ra = rb >= imm ?|\newline\verb|1 : 0|	\\ \hline
teq	&\verb|101 1000|	&test if equal			&\verb|ra = ra == rb ?|\newline\verb|1 : 0|		&\verb|ra = rb == imm ?|\newline\verb|1 : 0|	\\ \hline
tne	&\verb|101 1001|	&test if not equal		&\verb|ra = ra != rb ?|\newline\verb|1 : 0|		&\verb|ra = rb != imm ?|\newline\verb|1 : 0|	\\ \hline
lsl	&\verb|001 1000|	&shift left			&\verb|ra = ra << rb|			&\verb|ra = rb << imm|		\\ \hline
lsr	&\verb|001 1010|	&shift right			&\verb|ra = ra >>> rb|			&\verb|ra = rb >>> imm|		\\ \hline
asr	&\verb|001 1011|	&arithmetic shift right		&\verb|ra = ra >> rb|			&\verb|ra = rb >> imm|		\\ \hline
ldw	&\verb|010 0000|	&load word			&\verb|ra = mem[rb]|			&\verb|ra = mem[rb + imm]|	\\ \hline
ldb	&\verb|010 0010|	&load byte			&\verb|ra = mem[rb]|			&\verb|ra = mem[rb + imm]|	\\ \hline
lbu	&\verb|010 0011|	&load byte unsigned		&\verb|ra = mem[rb]|			&\verb|ra = mem[rb + imm]|	\\ \hline
stw	&\verb|010 0100|	&store word			&\verb|mem[rb] = ra|			&\verb|mem[rb + imm] = ra|	\\ \hline
stb	&\verb|010 0110|	&store byte			&\verb|mem[rb] = ra|			&\verb|mem[rb + imm] = ra|	\\ \hline
blt	&\verb|100 0000|	&branch if less than		&\verb|pc = ra < 0 ?|\newline\verb|rb : pc + 2|		&\verb|pc = ra < rb ?|\newline\verb|imm : pc + 2|	\\ \hline
bge	&\verb|100 0001|	&branch if greater or equal	&\verb|pc = ra >= 0 ?|\newline\verb|rb : pc + 2|	&\verb|pc = ra >= rb ?|\newline\verb|imm : pc + 2|	\\ \hline
bbl	&\verb|100 0100|	&branch if below		&\verb|pc = ra < 0 ?|\newline\verb|rb : pc + 2|		&\verb|pc = ra < rb ?|\newline\verb|imm : pc + 2|	\\ \hline
bae	&\verb|100 0101|	&branch if above or equal	&\verb|pc = ra >= 0 ?|\newline\verb|rb : pc + 2|	&\verb|pc = ra >= rb ?|\newline\verb|imm : pc + 2|	\\ \hline
beq	&\verb|100 1000|	&branch if equal		&\verb|pc = ra == 0 ?|\newline\verb|rb : pc + 2|	&\verb|pc = ra == rb ?|\newline\verb|imm : pc + 2|	\\ \hline
bne	&\verb|100 1001|	&branch if not equal		&\verb|pc = ra != 0 ?|\newline\verb|rb : pc + 2|	&\verb|pc = ra != rb ?|\newline\verb|imm : pc + 2|	\\ \hline
hlt	&\verb|111 1100|	&halt execution (simulador)	& 	&	\\ \hline
\end{tabular}
\end{table}

\subsection{Lógica e aritmética}

\subsubsection{AND - logical and (bitwise logical product)}
Realiza o produto lógico de dois valores e armazena o resultado em um
registrador. 

\begin{table}[htb!]
\centering
\begin{tabular}{|p{2cm}|p{2cm}|p{2cm}|p{2cm}|p{2cm}|}
\hline
$I_{<15:13>}$ & $I_{<12:9>}$ & $I_{<8>}$ & $I_{<7:4>}$ & $I_{<3:0>}$  \\ \hline
\textit{Opcode} & \textit{Opcode 2} & \textit{Imm} & \textit{ra} & \textit{rb} \\ \hline
\texttt{0 0 0} & \texttt{0 0 0 0} & \texttt{i} & \texttt{r r r r} & \texttt{r r r r} \\ \hline
\end{tabular}
\end{table}

\begin{itemize}
\item AND ra, rb
\subitem \texttt{GPR[ra] $\leftarrow$ GPR[ra] and GPR[rb]}
\item AND ra, rb, Immediate
\subitem \texttt{GPR[ra] $\leftarrow$ GPR[rb] and Immediate}
\end{itemize}

\subsubsection{OR - logical or (bitwise logical sum)}
Realiza a soma lógica de dois valores e armazena o resultado em um
registrador.

\begin{table}[htb!]
\centering
\begin{tabular}{|p{2cm}|p{2cm}|p{2cm}|p{2cm}|p{2cm}|}
\hline
$I_{<15:13>}$ & $I_{<12:9>}$ & $I_{<8>}$ & $I_{<7:4>}$ & $I_{<3:0>}$  \\ \hline
\textit{Opcode} & \textit{Opcode 2} & \textit{Imm} & \textit{ra} & \textit{rb} \\ \hline
\texttt{0 0 0} & \texttt{0 0 1 0} & \texttt{i} & \texttt{r r r r} & \texttt{r r r r} \\ \hline
\end{tabular}
\end{table}

\begin{itemize}
\item OR ra, rb
\subitem \texttt{GPR[ra] $\leftarrow$ GPR[ra] or GPR[rb]}
\item OR ra, rb, Immediate
\subitem \texttt{GPR[ra] $\leftarrow$ GPR[rb] or Immediate}
\end{itemize}

\subsubsection{XOR - logical exclusive or (bitwise logical difference)}
Realiza a diferença lógica de dois valores e armazena o resultado em um
registrador.

\begin{table}[htb!]
\centering
\begin{tabular}{|p{2cm}|p{2cm}|p{2cm}|p{2cm}|p{2cm}|}
\hline
$I_{<15:13>}$ & $I_{<12:9>}$ & $I_{<8>}$ & $I_{<7:4>}$ & $I_{<3:0>}$  \\ \hline
\textit{Opcode} & \textit{Opcode 2} & \textit{Imm} & \textit{ra} & \textit{rb} \\ \hline
\texttt{0 0 0} & \texttt{0 0 1 1} & \texttt{i} & \texttt{r r r r} & \texttt{r r r r} \\ \hline
\end{tabular}
\end{table}

\begin{itemize}
\item XOR ra, rb
\subitem \texttt{GPR[ra] $\leftarrow$ GPR[ra] xor GPR[rb]}
\item XOR ra, rb, Immediate
\subitem \texttt{GPR[ra] $\leftarrow$ GPR[rb] xor Immediate}
\end{itemize}

\subsubsection{ADD - arithmetic add}
Soma dois valores e armazena o resultado em um registrador.

\begin{table}[htb!]
\centering
\begin{tabular}{|p{2cm}|p{2cm}|p{2cm}|p{2cm}|p{2cm}|}
\hline
$I_{<15:13>}$ & $I_{<12:9>}$ & $I_{<8>}$ & $I_{<7:4>}$ & $I_{<3:0>}$  \\ \hline
\textit{Opcode} & \textit{Opcode 2} & \textit{Imm} & \textit{ra} & \textit{rb} \\ \hline
\texttt{0 0 0} & \texttt{0 1 0 0} & \texttt{i} & \texttt{r r r r} & \texttt{r r r r} \\ \hline
\end{tabular}
\end{table}

\begin{itemize}
\item ADD ra, rb
\subitem \texttt{GPR[ra] $\leftarrow$ GPR[ra] + GPR[rb]}
\item ADD ra, rb, Immediate
\subitem \texttt{GPR[ra] $\leftarrow$ GPR[rb] + Immediate}
\end{itemize}

\subsubsection{SUB - arithmetic subtract}
Subtrai dois valores e armazena o resultado em um registrador.

\begin{table}[htb!]
\centering
\begin{tabular}{|p{2cm}|p{2cm}|p{2cm}|p{2cm}|p{2cm}|}
\hline
$I_{<15:13>}$ & $I_{<12:9>}$ & $I_{<8>}$ & $I_{<7:4>}$ & $I_{<3:0>}$  \\ \hline
\textit{Opcode} & \textit{Opcode 2} & \textit{Imm} & \textit{ra} & \textit{rb} \\ \hline
\texttt{0 0 0} & \texttt{0 1 0 1} & \texttt{i} & \texttt{r r r r} & \texttt{r r r r} \\ \hline
\end{tabular}
\end{table}

\begin{itemize}
\item SUB ra, rb
\subitem \texttt{GPR[ra] $\leftarrow$ GPR[ra] - GPR[rb]}
\item SUB ra, rb, Immediate
\subitem \texttt{GPR[ra] $\leftarrow$ GPR[rb] - Immediate}
\end{itemize}


\subsection{Comparação}

\subsubsection{TLT - test if less than}
Testa se o primeiro valor é menor que o segundo. O teste resulta em
verdadeiro (1) ou falso (0), sendo o resultado armazenado em um
registrador.

\begin{table}[htb!]
\centering
\begin{tabular}{|p{2cm}|p{2cm}|p{2cm}|p{2cm}|p{2cm}|}
\hline
$I_{<15:13>}$ & $I_{<12:9>}$ & $I_{<8>}$ & $I_{<7:4>}$ & $I_{<3:0>}$  \\ \hline
\textit{Opcode} & \textit{Opcode 2} & \textit{Imm} & \textit{ra} & \textit{rb} \\ \hline
\texttt{1 0 1} & \texttt{0 0 0 0} & \texttt{i} & \texttt{r r r r} & \texttt{r r r r} \\ \hline
\end{tabular}
\end{table}

\begin{itemize}
\item TLT ra, rb
\subitem \texttt{GPR[ra] $\leftarrow$ GPR[ra] < GPR[rb] ? 1 : 0}
\item TLT ra, rb, Immediate
\subitem \texttt{GPR[ra] $\leftarrow$ GPR[rb] < Immediate ? 1 : 0}
\end{itemize}

\subsubsection{TGE - test if greater or equal}
Testa se o primeiro valor é maior ou igual ao o segundo. O teste resulta
em verdadeiro (1) ou falso (0), sendo o resultado armazenado em um
registrador.

\begin{table}[htb!]
\centering
\begin{tabular}{|p{2cm}|p{2cm}|p{2cm}|p{2cm}|p{2cm}|}
\hline
$I_{<15:13>}$ & $I_{<12:9>}$ & $I_{<8>}$ & $I_{<7:4>}$ & $I_{<3:0>}$  \\ \hline
\textit{Opcode} & \textit{Opcode 2} & \textit{Imm} & \textit{ra} & \textit{rb} \\ \hline
\texttt{1 0 1} & \texttt{0 0 0 1} & \texttt{i} & \texttt{r r r r} & \texttt{r r r r} \\ \hline
\end{tabular}
\end{table}

\begin{itemize}
\item TGE ra, rb
\subitem \texttt{GPR[ra] $\leftarrow$ GPR[ra] >= GPR[rb] ? 1 : 0}
\item TGE ra, rb, Immediate
\subitem \texttt{GPR[ra] $\leftarrow$ GPR[rb] >= Immediate ? 1 : 0}
\end{itemize}

\subsubsection{TBL - test if below}
Testa se o primeiro valor é menor que o segundo (valores não sinalizados).
O teste resulta em verdadeiro (1) ou falso (0), sendo o resultado
armazenado em um registrador.

\begin{table}[htb!]
\centering
\begin{tabular}{|p{2cm}|p{2cm}|p{2cm}|p{2cm}|p{2cm}|}
\hline
$I_{<15:13>}$ & $I_{<12:9>}$ & $I_{<8>}$ & $I_{<7:4>}$ & $I_{<3:0>}$  \\ \hline
\textit{Opcode} & \textit{Opcode 2} & \textit{Imm} & \textit{ra} & \textit{rb} \\ \hline
\texttt{1 0 1} & \texttt{0 1 0 0} & \texttt{i} & \texttt{r r r r} & \texttt{r r r r} \\ \hline
\end{tabular}
\end{table}

\begin{itemize}
\item TBL ra, rb
\subitem \texttt{GPR[ra] $\leftarrow$ GPR[ra] < GPR[rb] ? 1 : 0}
\item TBL ra, rb, Immediate
\subitem \texttt{GPR[ra] $\leftarrow$ GPR[rb] < Immediate ? 1 : 0}
\end{itemize}

\subsubsection{TAE - test if above or equal}
Testa se o primeiro valor é maior ou igual ao segundo (valores não
sinalizados). O teste resulta em verdadeiro (1) ou falso (0), sendo o
resultado armazenado em um registrador.

\begin{table}[htb!]
\centering
\begin{tabular}{|p{2cm}|p{2cm}|p{2cm}|p{2cm}|p{2cm}|}
\hline
$I_{<15:13>}$ & $I_{<12:9>}$ & $I_{<8>}$ & $I_{<7:4>}$ & $I_{<3:0>}$  \\ \hline
\textit{Opcode} & \textit{Opcode 2} & \textit{Imm} & \textit{ra} & \textit{rb} \\ \hline
\texttt{1 0 1} & \texttt{0 1 0 1} & \texttt{i} & \texttt{r r r r} & \texttt{r r r r} \\ \hline
\end{tabular}
\end{table}

\begin{itemize}
\item TAE ra, rb
\subitem \texttt{GPR[ra] $\leftarrow$ GPR[ra] >= GPR[rb] ? 1 : 0}
\item TAE ra, rb, Immediate
\subitem \texttt{GPR[ra] $\leftarrow$ GPR[rb] >= Immediate ? 1 : 0}
\end{itemize}

\subsubsection{TEQ - test if equal}
Testa se o primeiro valor é igual ao segundo. O teste resulta em
verdadeiro (1) ou falso (0), sendo o resultado armazenado em um
registrador.

\begin{table}[htb!]
\centering
\begin{tabular}{|p{2cm}|p{2cm}|p{2cm}|p{2cm}|p{2cm}|}
\hline
$I_{<15:13>}$ & $I_{<12:9>}$ & $I_{<8>}$ & $I_{<7:4>}$ & $I_{<3:0>}$  \\ \hline
\textit{Opcode} & \textit{Opcode 2} & \textit{Imm} & \textit{ra} & \textit{rb} \\ \hline
\texttt{1 0 1} & \texttt{1 0 0 0} & \texttt{i} & \texttt{r r r r} & \texttt{r r r r} \\ \hline
\end{tabular}
\end{table}

\begin{itemize}
\item TEQ ra, rb
\subitem \texttt{GPR[ra] $\leftarrow$ GPR[ra] == GPR[rb] ? 1 : 0}
\item TEQ ra, rb, Immediate
\subitem \texttt{GPR[ra] $\leftarrow$ GPR[rb] == Immediate ? 1 : 0}
\end{itemize}

\subsubsection{TNE - test if not equal}
Testa se o primeiro valor é diferente do segundo. O teste resulta em
verdadeiro (1) ou falso (0), sendo o resultado armazenado em um
registrador.

\begin{table}[htb!]
\centering
\begin{tabular}{|p{2cm}|p{2cm}|p{2cm}|p{2cm}|p{2cm}|}
\hline
$I_{<15:13>}$ & $I_{<12:9>}$ & $I_{<8>}$ & $I_{<7:4>}$ & $I_{<3:0>}$  \\ \hline
\textit{Opcode} & \textit{Opcode 2} & \textit{Imm} & \textit{ra} & \textit{rb} \\ \hline
\texttt{1 0 1} & \texttt{1 0 0 1} & \texttt{i} & \texttt{r r r r} & \texttt{r r r r} \\ \hline
\end{tabular}
\end{table}

\begin{itemize}
\item TNE ra, rb
\subitem \texttt{GPR[ra] $\leftarrow$ GPR[ra] != GPR[rb] ? 1 : 0}
\item TNE ra, rb, Immediate
\subitem \texttt{GPR[ra] $\leftarrow$ GPR[rb] != Immediate ? 1 : 0}
\end{itemize}


\subsection{Deslocamento}

\subsubsection{LSL - logical shift left}
Realiza a o deslocamento lógico à esquerda e armazena o resultado em um
registrador. Zeros são inseridos na parte menos significativa do
resultado.

\begin{table}[htb!]
\centering
\begin{tabular}{|p{2cm}|p{2cm}|p{2cm}|p{2cm}|p{2cm}|}
\hline
$I_{<15:13>}$ & $I_{<12:9>}$ & $I_{<8>}$ & $I_{<7:4>}$ & $I_{<3:0>}$  \\ \hline
\textit{Opcode} & \textit{Opcode 2} & \textit{Imm} & \textit{ra} & \textit{rb} \\ \hline
\texttt{0 0 1} & \texttt{1 0 0 0} & \texttt{i} & \texttt{r r r r} & \texttt{r r r r} \\ \hline
\end{tabular}
\end{table}

\begin{itemize}
\item LSL ra, rb
\subitem \texttt{GPR[ra] $\leftarrow$ GPR[ra] << GPR[rb]}
\item LSL ra, rb, Immediate
\subitem \texttt{GPR[ra] $\leftarrow$ GPR[rb] << Immediate}
\end{itemize}

\subsubsection{LSR - logical shift right}

Realiza a o deslocamento lógico à direita e armazena o resultado em um
registrador. Zeros são inseridos na parte mais significativa do resultado.

\begin{table}[htb!]
\centering
\begin{tabular}{|p{2cm}|p{2cm}|p{2cm}|p{2cm}|p{2cm}|}
\hline
$I_{<15:13>}$ & $I_{<12:9>}$ & $I_{<8>}$ & $I_{<7:4>}$ & $I_{<3:0>}$  \\ \hline
\textit{Opcode} & \textit{Opcode 2} & \textit{Imm} & \textit{ra} & \textit{rb} \\ \hline
\texttt{0 0 1} & \texttt{1 0 1 0} & \texttt{i} & \texttt{r r r r} & \texttt{r r r r} \\ \hline
\end{tabular}
\end{table}

\begin{itemize}
\item LSR ra, rb
\subitem \texttt{GPR[ra] $\leftarrow$ GPR[ra] >>> GPR[rb]}
\item LSR ra, rb, Immediate
\subitem \texttt{GPR[ra] $\leftarrow$ GPR[rb] >>> Immediate}
\end{itemize}

\subsubsection{ASR - arithmetic shift right}

Realiza a o deslocamento aritmético à direita e armazena o resultado em
um registrador. Zeros ou uns são inseridos na parte mais significativa
do resultado, preservando o sinal.

\begin{table}[htb!]
\centering
\begin{tabular}{|p{2cm}|p{2cm}|p{2cm}|p{2cm}|p{2cm}|}
\hline
$I_{<15:13>}$ & $I_{<12:9>}$ & $I_{<8>}$ & $I_{<7:4>}$ & $I_{<3:0>}$  \\ \hline
\textit{Opcode} & \textit{Opcode 2} & \textit{Imm} & \textit{ra} & \textit{rb} \\ \hline
\texttt{0 0 1} & \texttt{1 0 1 1} & \texttt{i} & \texttt{r r r r} & \texttt{r r r r} \\ \hline
\end{tabular}
\end{table}

\begin{itemize}
\item ASR ra, rb
\subitem \texttt{GPR[ra] $\leftarrow$ GPR[ra] >> GPR[rb]}
\item ASR ra, rb, Immediate
\subitem \texttt{GPR[ra] $\leftarrow$ GPR[rb] >> Immediate}
\end{itemize}


\subsection{Transferência de dados}

\subsubsection{LDW - load word}

Transfere uma palavra da memória para um registrador. O endereço acessado
deve ser alinhado ao tamanho da palavra (16 bits).

\begin{table}[htb!]
\centering
\begin{tabular}{|p{2cm}|p{2cm}|p{2cm}|p{2cm}|p{2cm}|}
\hline
$I_{<15:13>}$ & $I_{<12:9>}$ & $I_{<8>}$ & $I_{<7:4>}$ & $I_{<3:0>}$  \\ \hline
\textit{Opcode} & \textit{Opcode 2} & \textit{Imm} & \textit{ra} & \textit{rb} \\ \hline
\texttt{0 1 0} & \texttt{0 0 0 0} & \texttt{i} & \texttt{r r r r} & \texttt{r r r r} \\ \hline
\end{tabular}
\end{table}

\begin{itemize}
\item LDW ra, rb
\subitem \texttt{GPR[ra] $\leftarrow$ MEM[rb]}
\item LDW ra, rb, Immediate
\subitem \texttt{GPR[ra] $\leftarrow$ MEM[rb + Immediate]}
\end{itemize}

\subsubsection{LDB - load byte}

Transfere um byte da memória para um registrador. O registrador de
destino que armazena o resultado tem o byte carregado na parte menos
significativa, sendo o sinal preservado (ocorre extensão de sinal).

\begin{table}[htb!]
\centering
\begin{tabular}{|p{2cm}|p{2cm}|p{2cm}|p{2cm}|p{2cm}|}
\hline
$I_{<15:13>}$ & $I_{<12:9>}$ & $I_{<8>}$ & $I_{<7:4>}$ & $I_{<3:0>}$  \\ \hline
\textit{Opcode} & \textit{Opcode 2} & \textit{Imm} & \textit{ra} & \textit{rb} \\ \hline
\texttt{0 1 0} & \texttt{0 0 1 0} & \texttt{i} & \texttt{r r r r} & \texttt{r r r r} \\ \hline
\end{tabular}
\end{table}

\begin{itemize}
\item LDB ra, rb
\subitem \texttt{GPR[ra] $\leftarrow$ SEXT(MEM[rb]\textsubscript{<7:0>})}
\item LDB ra, rb, Immediate
\subitem \texttt{GPR[ra] $\leftarrow$ SEXT(MEM[rb + Immediate]\textsubscript{<7:0>})}
\end{itemize}

\subsubsection{LBU - load byte unsigned}

Transfere um byte da memória para um registrador. O registrador de
destino que armazena o resultado tem o byte carregado na parte menos
significativa, e a parte mais significativa preenchida com zeros.

\begin{table}[htb!]
\centering
\begin{tabular}{|p{2cm}|p{2cm}|p{2cm}|p{2cm}|p{2cm}|}
\hline
$I_{<15:13>}$ & $I_{<12:9>}$ & $I_{<8>}$ & $I_{<7:4>}$ & $I_{<3:0>}$  \\ \hline
\textit{Opcode} & \textit{Opcode 2} & \textit{Imm} & \textit{ra} & \textit{rb} \\ \hline
\texttt{0 1 0} & \texttt{0 0 1 1} & \texttt{i} & \texttt{r r r r} & \texttt{r r r r} \\ \hline
\end{tabular}
\end{table}

\begin{itemize}
\item LBU ra, rb
\subitem \texttt{GPR[ra] $\leftarrow$ ZEXT(MEM[rb]\textsubscript{<7:0>})}
\item LBU ra, rb, Immediate
\subitem \texttt{GPR[ra] $\leftarrow$ ZEXT(MEM[rb + Immediate]\textsubscript{<7:0>})}
\end{itemize}

\subsubsection{STW - store word}

Transfere uma palavra para memória a partir de um registrador. O endereço
acessado deve ser alinhado ao tamanho da palavra (16 bits).

\begin{table}[htb!]
\centering
\begin{tabular}{|p{2cm}|p{2cm}|p{2cm}|p{2cm}|p{2cm}|}
\hline
$I_{<15:13>}$ & $I_{<12:9>}$ & $I_{<8>}$ & $I_{<7:4>}$ & $I_{<3:0>}$  \\ \hline
\textit{Opcode} & \textit{Opcode 2} & \textit{Imm} & \textit{ra} & \textit{rb} \\ \hline
\texttt{0 1 0} & \texttt{0 1 0 0} & \texttt{i} & \texttt{r r r r} & \texttt{r r r r} \\ \hline
\end{tabular}
\end{table}

\begin{itemize}
\item STW ra, rb
\subitem \texttt{MEM[rb] $\leftarrow$ GPR[ra]}
\item STW ra, rb, Immediate
\subitem \texttt{MEM[rb + Immediate] $\leftarrow$ GPR[ra]}
\end{itemize}

\subsubsection{STB - load byte}

Transfere um byte para memória a partir de um registrador. O registrador
de fonte tem o byte carregado da parte menos significativa.

\begin{table}[htb!]
\centering
\begin{tabular}{|p{2cm}|p{2cm}|p{2cm}|p{2cm}|p{2cm}|}
\hline
$I_{<15:13>}$ & $I_{<12:9>}$ & $I_{<8>}$ & $I_{<7:4>}$ & $I_{<3:0>}$  \\ \hline
\textit{Opcode} & \textit{Opcode 2} & \textit{Imm} & \textit{ra} & \textit{rb} \\ \hline
\texttt{0 1 0} & \texttt{0 1 1 0} & \texttt{i} & \texttt{r r r r} & \texttt{r r r r} \\ \hline
\end{tabular}
\end{table}

\begin{itemize}
\item STB ra, rb
\subitem \texttt{MEM[rb] $\leftarrow$ GPR[ra]\textsubscript{<7:0>}}
\item STB ra, rb, Immediate
\subitem \texttt{MEM[rb + Immediate] $\leftarrow$ GPR[ra]\textsubscript{<7:0>}}
\end{itemize}


\subsection{Transferência de controle}

\subsubsection{BLT - branch if less than}
Testa se o primeiro valor é menor que o segundo. Caso o teste seja
verdadeiro, o valor do PC é atualizado com o endereço alvo.

\begin{table}[htb!]
\centering
\begin{tabular}{|p{2cm}|p{2cm}|p{2cm}|p{2cm}|p{2cm}|}
\hline
$I_{<15:13>}$ & $I_{<12:9>}$ & $I_{<8>}$ & $I_{<7:4>}$ & $I_{<3:0>}$  \\ \hline
\textit{Opcode} & \textit{Opcode 2} & \textit{Imm} & \textit{ra} & \textit{rb} \\ \hline
\texttt{1 0 0} & \texttt{0 0 0 0} & \texttt{i} & \texttt{r r r r} & \texttt{r r r r} \\ \hline
\end{tabular}
\end{table}

\begin{itemize}
\item BLT ra, rb
\subitem \texttt{PC $\leftarrow$ GPR[ra] < 0 ? GPR[rb] : PC + 2}
\item BLT ra, rb, Immediate
\subitem \texttt{PC $\leftarrow$ GPR[ra] < GPR[rb] ? Immediate :  PC + 2}
\end{itemize}

\subsubsection{BGE - branch if greater or equal}
Testa se o primeiro valor é maior ou igual ao segundo. Caso o teste
seja verdadeiro, o valor do PC é atualizado com o endereço alvo.

\begin{table}[htb!]
\centering
\begin{tabular}{|p{2cm}|p{2cm}|p{2cm}|p{2cm}|p{2cm}|}
\hline
$I_{<15:13>}$ & $I_{<12:9>}$ & $I_{<8>}$ & $I_{<7:4>}$ & $I_{<3:0>}$  \\ \hline
\textit{Opcode} & \textit{Opcode 2} & \textit{Imm} & \textit{ra} & \textit{rb} \\ \hline
\texttt{1 0 0} & \texttt{0 0 0 1} & \texttt{i} & \texttt{r r r r} & \texttt{r r r r} \\ \hline
\end{tabular}
\end{table}

\begin{itemize}
\item BGE ra, rb
\subitem \texttt{PC $\leftarrow$ GPR[ra] >= 0 ? GPR[rb] : PC + 2}
\item BGE ra, rb, Immediate
\subitem \texttt{PC $\leftarrow$ GPR[ra] >= GPR[rb] ? Immediate :  PC + 2}
\end{itemize}

\subsubsection{BBL - branch if below}
Testa se o primeiro valor é menor que o segundo (valores não sinalizados).
Caso o teste seja verdadeiro, o valor do PC é atualizado com o endereço
alvo.

\begin{table}[htb!]
\centering
\begin{tabular}{|p{2cm}|p{2cm}|p{2cm}|p{2cm}|p{2cm}|}
\hline
$I_{<15:13>}$ & $I_{<12:9>}$ & $I_{<8>}$ & $I_{<7:4>}$ & $I_{<3:0>}$  \\ \hline
\textit{Opcode} & \textit{Opcode 2} & \textit{Imm} & \textit{ra} & \textit{rb} \\ \hline
\texttt{1 0 0} & \texttt{0 1 0 0} & \texttt{i} & \texttt{r r r r} & \texttt{r r r r} \\ \hline
\end{tabular}
\end{table}

\begin{itemize}
\item BBL ra, rb
\subitem \texttt{PC $\leftarrow$ GPR[ra] < 0 ? GPR[rb] : PC + 2}
\item BBL ra, rb, Immediate
\subitem \texttt{PC $\leftarrow$ GPR[ra] < GPR[rb] ? Immediate :  PC + 2}
\end{itemize}

\subsubsection{BAE - branch if above or equal}
Testa se o primeiro valor é maior ou igual ao segundo (valores não
sinalizados). Caso o teste seja verdadeiro, o valor do PC é atualizado
com o endereço alvo.

\begin{table}[htb!]
\centering
\begin{tabular}{|p{2cm}|p{2cm}|p{2cm}|p{2cm}|p{2cm}|}
\hline
$I_{<15:13>}$ & $I_{<12:9>}$ & $I_{<8>}$ & $I_{<7:4>}$ & $I_{<3:0>}$  \\ \hline
\textit{Opcode} & \textit{Opcode 2} & \textit{Imm} & \textit{ra} & \textit{rb} \\ \hline
\texttt{1 0 0} & \texttt{0 1 0 1} & \texttt{i} & \texttt{r r r r} & \texttt{r r r r} \\ \hline
\end{tabular}
\end{table}

\begin{itemize}
\item BAE ra, rb
\subitem \texttt{PC $\leftarrow$ GPR[ra] >= 0 ? GPR[rb] : PC + 2}
\item BAE ra, rb, Immediate
\subitem \texttt{PC $\leftarrow$ GPR[ra] >= GPR[rb] ? Immediate :  PC + 2}
\end{itemize}

\subsubsection{BEQ - branch if equal}
Testa se o primeiro valor é igual ao segundo. Caso o teste seja
verdadeiro, o valor do PC é atualizado com o endereço alvo.

\begin{table}[htb!]
\centering
\begin{tabular}{|p{2cm}|p{2cm}|p{2cm}|p{2cm}|p{2cm}|}
\hline
$I_{<15:13>}$ & $I_{<12:9>}$ & $I_{<8>}$ & $I_{<7:4>}$ & $I_{<3:0>}$  \\ \hline
\textit{Opcode} & \textit{Opcode 2} & \textit{Imm} & \textit{ra} & \textit{rb} \\ \hline
\texttt{1 0 0} & \texttt{1 0 0 0} & \texttt{i} & \texttt{r r r r} & \texttt{r r r r} \\ \hline
\end{tabular}
\end{table}

\begin{itemize}
\item BEQ ra, rb
\subitem \texttt{PC $\leftarrow$ GPR[ra] == 0 ? GPR[rb] : PC + 2}
\item BEQ ra, rb, Immediate
\subitem \texttt{PC $\leftarrow$ GPR[ra] == GPR[rb] ? Immediate :  PC + 2}
\end{itemize}

\subsubsection{BNE - branch if not equal}
Testa se o primeiro valor é diferente do segundo. Caso o teste
seja verdadeiro, o valor do PC é atualizado com o endereço alvo.

\begin{table}[htb!]
\centering
\begin{tabular}{|p{2cm}|p{2cm}|p{2cm}|p{2cm}|p{2cm}|}
\hline
$I_{<15:13>}$ & $I_{<12:9>}$ & $I_{<8>}$ & $I_{<7:4>}$ & $I_{<3:0>}$  \\ \hline
\textit{Opcode} & \textit{Opcode 2} & \textit{Imm} & \textit{ra} & \textit{rb} \\ \hline
\texttt{1 0 0} & \texttt{1 0 0 1} & \texttt{i} & \texttt{r r r r} & \texttt{r r r r} \\ \hline
\end{tabular}
\end{table}

\begin{itemize}
\item BNE ra, rb
\subitem \texttt{PC $\leftarrow$ GPR[ra] != 0 ? GPR[rb] : PC + 2}
\item BNE ra, rb, Immediate
\subitem \texttt{PC $\leftarrow$ GPR[ra] != GPR[rb] ? Immediate :  PC + 2}
\end{itemize}


\section{Modos de endereçamento}

\subsection{Registrador}

No modo de endereçamento a registrador, dois registradores são referenciados
no formato de instrução. Os dois registradores são utilizados como
operandos, sendo o primeiro também utilizado para o armazenamento do
resultado da operação.

O tamanho total da instrução é de 2 bytes (1 palavra). A palavra contém
a informação referente à instrução e os operandos. Exemplos:

\begin{table}[htb!]
\centering
\begin{tabular}{|p{5.0cm}|p{8.0cm}|}
\hline
\bf{Operação} & \bf{Significado} \\ \hline \hline
\texttt{add r1,r2} & \verb|r1 = r1 + r2| \\ \hline
\texttt{xor r1,r1} & \verb|r1 = 0| \\ \hline
\texttt{asr r1,r2} & \verb|r1 = r1 >> r2| \\ \hline
\texttt{teq r1,r2} & \verb|r1 = r1 == r2 ? 1 : 0| \\ \hline
\end{tabular}
\end{table}

\subsection{Registrador com imediato}

No modo de endereçamento registrador com imediato, dois registradores
são referenciados no formato de instrução, além de um valor literal. O
segundo registrador e o valor imediato são utilizados como operandos, 
sendo o primeiro registrador utilizado para o armazenamento do resultado
da operação. Caso \textit{r0} for referenciado como o primeiro registrador,
o resultado da operação será descartado (exceto em desvios condicionais).
Caso \textit{r0} for referenciado como segundo registrador, a instrução
poderá resultar em uma carga do valor imediato no primeiro registrador,
o que é útil para a carga de constantes ou valores literais. Caso o valor
imediato for zero, a instrução poderá resultar na cópia do valor de um
registrador para outro.

O tamanho total da instrução é de 4 bytes (2 palavras). A primeira 
palavra contém a informação referente à instrução, o registrador para
o armazenamento do resultado e o primeiro operando. A segunda palavra
contém o valor imediato, utilizado como segundo operando. Exemplos:

\begin{table}[htb!]
\centering
\begin{tabular}{|p{5.0cm}|p{8.0cm}|}
\hline
\bf{Operação} & \bf{Significado} \\ \hline \hline
\texttt{and r1,r2,123} & \verb|r1 = r2 and 123| \\ \hline
\texttt{add r1,r1,1} & \verb|r1 = r1 + 1| \\ \hline
\texttt{xor r1,r1,-1} & \verb|r1 = not r1| \\ \hline
\texttt{lsl r1,r2,3} & \verb|r1 = r2 << 3| \\ \hline
\texttt{add r1,r0,12345} & \verb|r1 = 12345| \\ \hline
\texttt{add r1,r2,0} & \verb|r1 = r2| \\ \hline
\texttt{beq r0,r0,12345} & \verb|pc = 12345| \\ \hline
\texttt{bge r1,r2,12345} & \verb|pc = r1 >= r2 ? 12345 : pc + 2| \\ \hline
\texttt{tne r1,r2,3} & \verb|r1 = r2 != 3 ? 1 : 0| \\ \hline
\end{tabular}
\end{table}

\subsection{Indireto a registrador (base)}

No modo de endereçamento indireto a registrador, dois registradores são
referenciados no formato de instrução. O primeiro registrador é utilizado
como a origem, destino ou como operando e o segundo registrador é
utilizado como o endereço base referenciado. O operando zero pode ser
utilizado de forma implícita em desvios condicionais. Caso \textit{r0}
for referenciado como o primeiro registrador, em operações de carga o
valor será descartado, em operações de armazenamento o valor zero será
utilizado como origem e em operações de desvio este poderá realizado ou
não (comparação com zero).

O tamanho total da instrução é de 2 bytes (1 palavra). A palavra contém
a informação referente à instrução, a origem (ou destino / operando) e o
endereço base. Exemplos:

\begin{table}[htb!]
\centering
\begin{tabular}{|p{5.0cm}|p{8.0cm}|}
\hline
\bf{Operação} & \bf{Significado} \\ \hline \hline
\texttt{ldw r1,r2} & \verb|r1 = MEM[r2]| \\ \hline
\texttt{stw r1,r2} & \verb|MEM[r2] = r1| \\ \hline
\texttt{beq r0,r1} & \verb|pc = r1| \\ \hline
\texttt{blt r1,r2} & \verb|pc = r1 < 0 ? r2 : pc + 2| \\ \hline
\end{tabular}
\end{table}

\subsection{Indireto a registrador com deslocamento (base + deslocamento)}

No modo de endereçamento indireto a registrador com deslocamento, dois
registradores são referenciados no formato de instrução, além de um valor
imediato. O primeiro registrador é utilizado como origem ou destino
e o segundo registrador e o valor imediato são utilizados como operandos
para uma operação de soma, efetivamente calculando o endereço base + 
deslocamento. Caso o registrador base for \textit{r0}, o endereço é
calculado de forma absoluta. Caso o valor imediato for zero, o resultado
é o mesmo da operação utilizando o modo de endereçamento anterior
(indireto a registrador). 

O tamanho total da instrução é de 4 bytes (2 palavras). A primeira
palavra contém a informação referente à instrução, a origem (ou destino
/ operando) e o endereço base. A segunda palavra contém o valor imediato,
utilizado como deslocamento a partir da base. Exemplos:

\begin{table}[htb!]
\centering
\begin{tabular}{|p{5.0cm}|p{8.0cm}|}
\hline
\bf{Operação} & \bf{Significado} \\ \hline \hline
\texttt{stw r1,r2,12} & \verb|MEM[r2 + 12] = r1| \\ \hline
\texttt{ldb r1,r0,1000} & \verb|r1 = MEM[1000]| \\ \hline
\texttt{ldw r1,r2,-16} & \verb|r1 = MEM[r2 - 16]| \\ \hline
\end{tabular}
\end{table}


\chapter{Linguagem de montagem}

A linguagem de montagem da arquitetura TRM possui uma série de elementos
em sua sintaxe. Tais elementos são usados para a implementação de
programas. A sintaxe da linguagem será apresentada brevemente, assim como
alguns exemplos, os quais ilustram os principais usos da linguagem.

\section{Sintaxe}

A sintaxe da linguagem de montagem é bastante simples, contendo apenas
os elementos essenciais para criação de programas. Não é necessário
definir regiões separadas para código e dados\footnote{A arquitetura TRM
define um único espaço de endereços. No mesmo espaço de endereços
são armazenadas instruções e dados.} e também não é necessário o uso 
de diretivas tradicionalmente utilizadas em montadores para outras
arquiteturas.

\subsection{Comentários}

Comentários podem ser adicionados ao código de montagem, de acordo com
algumas regras. O caracter ';' é utilizado para inserção de comentários,
sendo o restante da linha o conteúdo do comentário. Após a inserção do
caracter ';', o restante da linha será ignorado no processo de montagem.
A inclusão de comentários no código deve ser feita apenas com caracteres
da língua inglesa.

\subsection{Endereços}

Um endereço de memória é um identificador numérico único para uma
localização específica na memória do computador. Endereços são valores
que indicam a posição de objetos, como palavras contendo instruções ou
dados na memória. Endereços são utilizados para criação de referências a
objetos, de tal maneira que, caso a memória seja acessada nessa posição
(derreferência), têm-se o acesso ao conteúdo do endereço. 

\subsection{Referências (símbolos)}

Referências são nomes simbólicos, ou seja, uma abstração de endereços
de memória, que são utilizados pelos programas em linguagem de montagem
para que a ferramenta de montagem (montador) os resolva. Durante o
processo de montagem de um programa, as referências são substituídas por
seu valor numérico. Dessa forma, uma referência é um nome que se dá a
uma posição (ou endereço) de memória.

Referências são utilizadas para declaração de pontos específicos no código,
tais como destino de saltos, ponto de entrada em funções / subprogramas
e também como declarações de estruturas de dados, como variáveis e
vetores. O montador é responsável por resolver as referências,
permitindo que estas assumam no código objeto um valor numérico para
sua codificação nas instruções montadas.

O valor numérico de uma referência será gerado em função do espaço alocado
previamente a declaração da mesma. Esse espaço pode ser ocupado por
declarações de instruções ou estruturas de dados. A referência, em si,
não ocupa memória alguma\footnote{Uma exceção é a declaração de uma
variável contendo um símbolo (referência) para um ponto do programa.},
sendo apenas um símbolo no programa, ou seja, o valor do endereço
da próxima palavra na listagem do código objeto gerado pelo montador (no
caso de uma declaração) ou o valor do endereço de uma posição de memória
citada em uma instrução ou estrutura de dados. A seguir, é apresentado
um exemplo do uso de referências em um programa:

\begin{verbatim}
ref1
    (instruções)
    ...
ref2
    (instruções)
    ...
ref3
    (dados)
    ...
\end{verbatim}

No exemplo, são apresentados três símbolos (\textit{ref1}, \textit{ref2}
e \textit{ref3}). Referências devem ser declaradas com alinhamento à 
esquerda, sem tabulações ou espaços e sem um caracter finalizador (como
dois pontos, por exemplo).

\subsection{Instruções}

Instruções são comandos que correspondem diretamente a operações que a
CPU deve executar no contexto de um programa. Cada linha de um programa
em linguagem de montagem que realiza uma ação concreta é uma instrução.
Instruções são representações legíveis do que se tornará o código de
máquina após o processo de montagem.

Na sintaxe da linguagem de montagem, instruções são divididas em duas
partes:

\begin{enumerate}
\item Mnemônico - é uma abreviação que representa a operação a ser
realizada (por exemplo, '\textit{add}' para uma operação de soma).
\item Operandos - definem o contexto no qual a operação será realizada
(por exemplo, os registradores e valor imediato)\footnote{Os operandos
em uma instrução são separados por vírgula.}.
\end{enumerate}

A seguir, é apresentado um exemplo do uso de instruções em um programa.
Note que algumas instruções fazem referência à símbolos, sendo necessário
que o montador resolva as referências antes de realizar a montagem das
instruções.

\begin{verbatim}
ref1
    add r1,r2,123
    xor r3,r4
    blt r6,r1,ref1
    ...
ref2
    ldw r4,r0,ref3
    ...
ref3
    (dado)
    ...
\end{verbatim}

Valores numéricos utilizados para definição de operandos imediatos podem
ser especificados em base 10 (por exemplo: 123), base 8 (0123) ou base
16 (0x123). O valor imediato também pode ser um símbolo, sendo seu valor
numérico resolvido pelo montador.

\subsection{Variáveis}

Variáveis são declaradas por meio de rótulos, que definem uma referência
a um espaço em memória contendo a informação representada por um valor
numérico. A alocação e acesso à variáveis ocorre por meio da reserva
do espaço em memória para o armazenamento do conteúdo de uma variável,
a nomeação dessa variável (símbolo ou referência) e por fim instruções
que manipulam o conteúdo. A alocação ocorre em tempo de montagem e o
acesso é feito em tempo de execução, quando as instruções do programa
executam.

Dois tipos de dados são definidos para a declaração de variáveis, sendo
esses uma palavra (inteiro de 16 bits) e um byte (inteiro de 8 bits).
Como pode ser observado, valores numéricos para variáveis do tipo byte
são precedidos pelo caractere '\$'. A alocação de bytes como variáveis
é feita de forma alinhada\footnote{O alinhamento de endereços consiste
em alocar o objeto em um endereço que coincide com o início de uma
palavra, usando-se um byte de \textit{padding}, caso necessário.}.
Inteiros são acessados por meio das instruções \textit{ldw} e \textit{stw}
(para leitura e escrita na memória, respectivamente) e bytes por
instruções \textit{ldb}, \textit{lbu} e \textit{stb} (as duas primeiras
usadas para leitura e a última para escrita na memória).

\begin{verbatim}
ref1
    (instruções)
    ...
var1
    123
var2
    -456
var3
    $-1
\end{verbatim}

No exemplo é apresentada a sintaxe para a declaração de variáveis.
\textit{var1} declara uma variável do tipo inteiro com valor positivo e
\textit{var2} declara um inteiro com valor negativo. É possível realizar
comparações sinalizadas e não sinalizadas com os valores de variáveis,
após sua carga em registradores. \textit{var3} define uma variável do
tipo byte. O seu valor pode ser interpretado de duas formas, dependendo
da instrução utilizada em uma operação de carga. Por exemplo, se a 
instrução \textit{ldb} for utilizada para carga do valor, este
terá o sinal preservado (no caso, -1) e a faixa representável está entre
-128 e 127. Caso a instrução \textit{lbu} for utilizada, o valor carregado
não terá o sinal preservado (no caso, 255), sendo a faixa representável
entre 0 e 255. Valores numéricos utilizados para definição de variáveis
podem ser especificados em base 10 (por exemplo: 123), base 8 (0123) ou
base 16 (0x123).

\subsection{Arranjos}

Arranjos são declarados da mesma forma que variáveis, ou seja, por
rótulos que definem uma referência para o início de um espaço contíguo
em memória contendo itens de um mesmo tipo. Existem quatro tipos de
arranjos possíveis (unidimensionais), sendo esses:

\begin{enumerate}
\item Inteiro
\item Byte
\item Reserva de espaço
\item String
\end{enumerate}

Arranjos dos tipos 1 e 2 devem ter seus itens separados por espaço em
uma declaração. Em um arranjo de bytes (tipo 2), os itens são precedidos
por um caracter '\$'. Nos arranjos do tipo 1 e 2, é alocado um espaço 
contíguo de palavras ou bytes na memória, ou seja, cada item do arranjo
possui 16 bits (2 bytes) para o tipo inteiro ou 8 bits (1 byte) para o
tipo byte.

Nos arranjos do tipo 3 (reserva de espaço), é definido um tamanho em
bytes de espaço contíguo a ser reservado em memória. No tipo 4, uma
\textit{string} tem seus bytes alocados contiguamente em memória. A
alocação de memória nos tipos 2, 3 e 4 é feita por meio de bytes alocados
e alinhados em palavras. Por exemplo, no arranjo \textit{byte\_array1}
temos 5 itens, sendo um item adicional (zero) alocado para o alinhamento,
o que gera um total de 3 palavras (6 bytes). O mesmo é feito em arranjos
do tipo 3 e 4, caso necessário.

\begin{verbatim}
ref1
    (instruções)
    ...
int_array
    123 456 -789 12345 -55
byte_array1
    $12 $34 $-56 $3
byte_array2
    [30]
byte_array3
    "hello world!"
\end{verbatim}

É importante ressaltar que strings possuem um byte terminador implícito
(zero) ao final. Isso é feito para simplificar o processo de impressão e
manipulação de strings, o que é feito da mesma forma em linguagens como
C, por exemplo. Valores numéricos utilizados para definição de itens
ou tamanho de arranjos podem ser especificados em base 8, 10 ou 16.

\subsection{Código objeto e desmontagem do programa}

O código objeto é o resultado do processamento de um programa em
linguagem de montagem para um formato binário, que pode então ser
executado pela CPU. Na maioria dos casos, o código objeto poderá ser
agregado a outras partes de código objeto (como bibliotecas de funções,
por exemplo) para a construção de um programa executável. 

A união de diversos códigos objeto, juntamente com a informação de relocação,
formada por endereços em uma tabela de símbolos são utilizadas por uma
ferramenta de ligação, para a construção do executável final. O processo
de ligação de diversos códigos objeto foge o escopo desse documento, que
se limita ao processo de montagem e geração de código de máquina que pode
ser executado diretamente por um simulador da arquitetura.

Para ilustrar o processo de montagem e geração do código objeto, será
utilizado um pequeno programa, que declara duas variáveis (\textit{var\_a}
e \textit{var\_b}), carrega os seus valores em registradores e soma os
mesmos.

\begin{verbatim}
main
    ldw v0,zr,var_a
    ldw v1,zr,var_b
    add v0,v1
    hlt
var_a
    123
var_b
    456
\end{verbatim}

O código gerado pelo processo de montagem possui 16 bytes, e este é
apresentado abaixo. Na primeira coluna do código, temos os endereços de
memória onde cada palavra do código objeto será carregado. As palavras
podem ser instruções ou dados. Na segunda coluna, temos o código objeto
propriamente dito, que são as palavras a serem carregadas na memória. Em
ambas as colunas, os valores são representados em base 16. Na terceira
coluna, é apresentado o processo inverso à montagem, conhecido como
\textit{desmontagem}. Essa informação é útil para análise do programa
após a montagem, podendo ser comparado ao código fonte.

\begin{verbatim}
0000 4150    (ldw r5,r0,12)
0002 000c
0004 4160    (ldw r6,r0,14)
0006 000e
0008 0856    (add r5,r6)
000a f800    (hlt r0,r0)
000c 007b
000e 01c8
\end{verbatim}

É importante observar que para algumas instruções são geradas duas
palavras, já para outras outras apenas uma. Isso é o resultado do uso de 
instruções com ou sem um campo imediado, o que depende
do modo de endereçamento usado na instrução específica. As variáveis
têm o seu espaço alocado e inicializado, não sendo apresentada uma
informação de desmontagem, assim como é feito com os valores literais.

\section{Exemplos}

\subsection{Inicialização de registradores}

\subsubsection{Constantes}

Constantes são necessárias em diversas situações em programas de
linguagem de montagem. Por exemplo, a carga de valores numéricos para
o processamento, referências dentro de um programa (símbolos), entre
outros. Na arquitetura TRM, não há uma instrução especial para a carga
de constantes, mas isso pode ser feito com o uso de instruções
\textit{add} ou \textit{or} em conjunto com o registrador \textit{zr}
(zero). Uma instrução \textit{xor} pode ser utilizada para limpar um
registrador de forma eficiente. Alguns exemplos:

\begin{verbatim}
    add v0,zr,1234              ; v0 = 1234
    or v1,zr,555                ; v1 = 555
    add v2,zr,0x222             ; v2 = 0x222
    add v3,zr,ref               ; v3 = &ref 
    xor v4,v4                   ; v4 = 0
\end{verbatim}

\subsubsection{Transferência valores entre registradores}

Para cópia valores entre registradores não há instruções especiais. É
possível utilizar instruções \textit{add} ou \textit{or} para isso:

\begin{verbatim}
    add v0,v1,0                 ; v0 = v1
    or v1,v2,0                  ; v1 = v2
\end{verbatim}

A troca de valores entre dois registradores (operação \textit{swap}) é
bastante eficiente, pois são utilizadas apenas instruções em formato
regular:

\begin{verbatim}
    ; v0 = 123, v1 = 456
    xor v0,v1
    xor v1,v0
    xor v0,v1
    ; v1 = 456, v0 = 123
\end{verbatim}

\subsubsection{Operações com três registradores}

Na arquitetura são definidas instruções com apenas dois registradores
ou dois registradores e uma constante. O padrão \textit{add r1,r2,r3},
existente em computadores como MIPS, RISC-V ou ARM pode ser emulado,
no entanto. Por exemplo, para implementar a instrução
\textit{sub v0,v1,v2} (onde \textit{v1} e \textit{v2} são operandos
fonte e \textit{v0} o destino) utilizam-se duas instruções:

\begin{verbatim}
    add v0,v1,0                 ; v0 = v1
    sub v0,v2                   ; v0 = v0 - v2
\end{verbatim}

\subsection{Complemento e inversão de sinal}

Não são definidas instruções específicas para operações de complemento
e inversão de sinal (complemento de 2). Para essas operações, utiliza-se
padrões simples. O complemento (operação \textit{not}) pode ser
sintetizado por uma operação \textit{xor} com o valor -1. Para o
complemento de 2, basta adicionar 1 ao resultado.

\begin{verbatim}
    xor v0,v0,-1                ; v0 = not v0 (one's complement)
    add v0,v0,1                 ; v0 = v0 + 1 (two's complement)
\end{verbatim}

\subsection{Controle de fluxo}

Tanto operações de seleção quanto laços condicionais podem ser implementados
com as mesmas operações. Testes podem ser realizados e seu resultado
(verdadeiro ou falso) armazenado em registradores ou por meio de desvios
condicionais. Nos exemplos serão utilizados desvios.

\subsubsection{Seleção}

As operações de seleção '$<$' (menor que), '$>=$' (maior ou igual),
'$==$' (igual) e '$!=$' (diferente) são simples de implementar, pois existem
instruções específicas para estes testes. Nos exemplos apresentados, o
desvio ocorre caso a condição seja verdadeira. Caso a condição seja
falsa, o programa segue (cláusula \textit{else}). Assume-se que os
valores \textit{a} e \textit{b} que estão sendo comparados, foram
carregados nos registradores \textit{v0} e \textit{v1} respectivamente.

\begin{multicols}{2}
\begin{itemize}
\item a $<$ b
\begin{verbatim}
    blt v0,v1,lessthan
else
    ...
lessthan
    ...
\end{verbatim}

\item a $>=$ b
\begin{verbatim}
    bge v0,v1,greaterequal
else
    ...
greaterequal
    ...
\end{verbatim}

\item a $==$ b
\begin{verbatim}
    beq v0,v1,equal
else
    ...
equal
    ...
\end{verbatim}

\item a $!=$ b
\begin{verbatim}
    bne v0,v1,notequal
else
    ...
notequal
    ...
\end{verbatim}
\end{itemize}
\end{multicols}

As operações '$>$' (maior que) e '$<=$' (menor ou igual) podem ser
sintetizadas pelas instruções \textit{blt} e \textit{bge}, bastando
inverter a ordem dos operandos.

\begin{multicols}{2}
\begin{itemize}
\item a $>$ b
\begin{verbatim}
    blt v1,v0,greaterthan
else
    ...
greaterthan
    ...
\end{verbatim}

\item a $<=$ b
\begin{verbatim}
    bge v1,v0,lessequal
else
    ...
lessequal
    ...
\end{verbatim}
\end{itemize}
\end{multicols}

\subsubsection{Laços}

Operações de repetição (laços) em linguagem de montagem possuem uma
estrutura semelhante a operações de seleção, com a diferença de que
o fluxo de execução será redirecionado a um ponto do código percorrido
anteriormente de maneira iterativa. Além da operação de repetição
(como um \textit{for}, \textit{while} ou \textit{do .. while}), muitas
vezes é necessário utilizar os comandos \textit{break} e \textit{continue}.
Em todos os casos, as estruturas utilizadas para essas operações são
semelhantes às apresentadas anteriormente.

Repetições incondicionais podem ser implementadas de acordo com o padrão
apresentado a seguir. No exemplo, assume-se que uma comparação por 
igualdade do registrador \textit{zr} (zero) com ele próprio seja sempre
verdadeira. Esse exemplo ilustra uma construção semelhante a um laço
\textit{while (1) \{ ... \}}.

\begin{verbatim}
loop
    ...                         ; loop body
    beq zr,zr,loop
endloop
\end{verbatim}

Repetições condicionais podem ser implementadas com um laço incondicional,
e internamente realizarem testes e condições de quebra do laço (break)
ou adiantamento do teste (continue). A implementação de um laço semelhante
a \textit{while (a $<$ b) \{ ... \}} é apresentrada a seguir. O teste
realizado para a quebra do laço é o oposto da condição de permanência.

\begin{verbatim}
loop
    bge v0,v1,endloop           ; if a >= b, break
    ...                         ; loop body
    beq zr,zr,loop
endloop
\end{verbatim}

Para um laço do tipo \textit{do \{ ... \} while (a $<$ b)}, basta que
o teste seja realizado imediatamente antes do desvio incondicional,
permitindo que o corpo do laço execute uma vez antes do teste ser feito.


\subsection{Variáveis e arranjos}

A arquitetura realiza o acesso à memória apenas com instruções de
transferência de dados (\textit{load} / \textit{store}). Dessa forma,
os operandos precisam ser trazidos da memória antes de serem realizadas
operações lógicas e aritméticas, comparações e desvios.

\subsubsection{Acesso à variáveis}

O acesso à variáveis pode ser feito diretamente por meio de instruções
\textit{ldw} e \textit{stw} (inteiros) ou \textit{ldb}, \textit{lbu} e
\textit{stb} (bytes). O acesso pode ser realizado por uma referência
ou por meio de um registrador com a referência.

\begin{verbatim}
    ...
    ldw v0,zr,var1              ; v0 = var1
    ...
    add v1,zr,var1              ; v1 = &var1
    ldw v0,v1                   ; v0 = *var1
    ...
    ldb v0,zr,var2              ; v0 = var2
    ...
    add v1,zr,var2              ; v1 = &var2
    ldb v0,v1                   ; v0 = *var2
    ...
var1
    0x999                       ; (integer)
var2
    $0xaa                       ; (byte)
\end{verbatim}

No exemplo apresentado abaixo, é feita a soma do valor de duas variáveis
e o resultado é armazenado em uma terceira (\textit{var\_c $=$ var\_a +
var\_b}). Ao todo, são realizados 3 acessos de dados à memória.

\lstinputlisting[caption=vars.s]{src/vars.s}

\subsubsection{Arranjos}

O acesso à arranjos pode ser feito diretamente por meio de instruções
\textit{ldw} e \textit{stw} (inteiros) ou \textit{ldb}, \textit{lbu} e
\textit{stb} (bytes). O acesso pode ser realizado por uma referência
em um registrador, somado a um deslocamento. Outra alternativa é utilizar
um endereço base fixo (referência ao início do arranjo), somado a um
deslocamento armazenado em um registador.

\begin{verbatim}
    ...
    add v3,zr,array             ; v3 = &array
    ldw v0,v3,0                 ; v0 = array[0]
    ldw v1,v3,2                 ; v1 = array[1]
    ...
    add v3,zr,array             ; v3 = &array
    ldw v0,v3                   ; v0 = *array
    add v3,v3,2                 ; v3 = &array++
    ldw v1,v3                   ; v1 = *array
    ...
    add v3,zr,0                 ; v3 = 0
    ldw v0,v3,array             ; v0 = array[0]
    add v3,v3,2                 ; v3 += 2
    ldw v1,v3,array             ; v1 = array[1]
    ...

array
    123 456 -789 111 222 -333   ; (array of integers)
\end{verbatim}

No exemplo abaixo é realizado o acesso a um vetor de inteiros, onde
cada elemento do vetor é acessado e modificado. É importante observar
que o vetor é indexado por meio do registrador \textit{v0}. Para
calcular o deslocamento na memória, é necessário multiplicar o índice
pelo tamanho do tipo (inteiro, 2 bytes), o que é feito com uma operação
\textit{lsl}. O limite de iterações do laço é dependente do número
de elementos no vetor.

\lstinputlisting[caption=array.s]{src/array.s}

O próximo exemplo realiza as mesmas operações, no entanto o acesso é
feito por meio de ponteiros. O deslocamento na memória é feito por meio
de aritmética de ponteiros, incrementando-se o endereço de memória por
duas posições (inteiro). O limite de iterações do laço é dependente do
endereço acessado.

\lstinputlisting[caption=arrayptr.s]{src/arrayptr.s}

No exemplo abaixo, é realizado o acesso a um arranjo de bytes, onde as
instruções \textit{ldb} e \textit{lbu} são utilizadas. Não é necessário
multiplicar o índice pelo tamanho do tipo, pois o acesso é do tipo byte.

\lstinputlisting[caption=arraybytes.s]{src/arraybytes.s}

\subsubsection{Strings}

O acesso a strings ocorre da mesma forma que um arranjo de bytes. No
exemplo abaixo, uma string é copiada para um espaço reservado (em
branco) de 30 bytes, e posteriormente a string copiada é impressa.

\lstinputlisting[caption=string.s]{src/string.s}

\subsection{A pilha}

A pilha tem como objetivo fazer uso da memória livre (não utilizada pelo
programa) como uma estrutura de dados gerenciável, sendo essa estrutura
essencial para a implementação de aplicações modulares. É por meio do
uso da pilha que torna-se possível implementar recursão, passagem de
um número maior de argumentos para funções entre outros.

O uso da memória como uma pilha pode ser generalizado em duas operações:
PUSH (armazena um item na pilha) e POP (retira um item da pilha). As
operações sobre a pilha utilizam um registrador especial para isso
(\textit{stack pointer} - \textit{r14} ou \textit{sp}). Antes que a
pilha possa ser utilizada, esse registrador deve ser inicializado com
uma referência ao \textit{topo da pilha}, que normalmente é o endereço
da última palavra da memória. Para implementar as operações PUSH e POP,
deve-se proceder da seguinte forma:

\begin{itemize}
\item PUSH - reservar um espaço na pilha, gravar o valor de um registrador
na pilha (no exemplo, \textit{v0});
\begin{verbatim}
sub sp,sp,2
stw v0,sp
\end{verbatim}
\item POP - ler o valor da pilha e gravar em um registrador, liberar um
espaço na pilha (no exemplo, \textit{v0}).
\begin{verbatim}
ldw v0,sp
add sp,sp,2
\end{verbatim}
\end{itemize}

Diversos itens podem ser colocados na pilha ou retirados de uma só vez.
Para isso, deve-se reservar o espaço na pilha e acessar a mesma para
as operações de leitura ou escrita por meio de um deslocamento. No
exemplo abaixo, 5 registradores são colocados na pilha de uma só vez,
usando-se operações de escrita após a alocação do espaço:

\begin{verbatim}
    sub sp,sp,10
    stw v0,sp,0
    stw v1,sp,2
    stw v2,sp,4
    stw v3,sp,6
    stw v4,sp,8
\end{verbatim}

E os mesmos podem ser retirados da mesma forma, lendo-se os valores da
pilha e desalocando o espaço.

\begin{verbatim}
    ldw v4,sp,8
    ldw v3,sp,6
    ldw v2,sp,4
    ldw v1,sp,2
    ldw v0,sp,0
    add sp,sp,10
\end{verbatim}

O programador é responsável por realizar esse gerenciamento, assim como
lembrar da ordem dos itens armazenados na pilha. A pilha pode ser usada,
entre outras coisas, para auxiliar na implementação de programas
modulares, usando-se chamadas de função.

\subsection{Ponteiro de quadro}

O ponteiro de quadro possui como objetivo permitir o acesso rápido a
variáveis locais, no contexto de uma função e permitir um uso mais
flexível do ponteiro de pilha. O ponteiro de quadro (\textit{frame
pointer} - \textit{r13}, \textit{fp} ou \textit{v8}) não precisa ser
utilizado para essa finalidade, podendo ser apenas mais um registrador
disponível ao programador, ou seja, seu uso é opcional.

Se for necessário alocar uma região de memória na pilha para uso no
contexto de uma função, pode-se configurar o registrador \textit{fp}
com o valor de \textit{sp} antes dessa alocação. Dessa maneira, é possível
acessar os dados por meio de uma referência que independe do valor de
\textit{sp}, que pode estar sendo utilizado para o processamento de
valores intermediários que residem na pilha. Caso for utilizado, esse
também é uma forma eficiente de desalocar dados da própria pilha, uma
vez que pode-se copiar o valor do registrador \textit{fp} para o
\textit{sp} no término de uma função.

\subsection{Convenções para chamadas de função}

Algumas convenções são definidas para chamadas de função. Essas
convenções devem ser seguidas para que haja compatibilidade na
implementação de aplicações e bibliotecas de funções.

\begin{itemize}
\item \textit{a0 - a3} são registradores de argumento e temporários e
são usados para passagem de parâmetros, operações de processamento e
retorno de valores.
\item \textit{lr} deve ser salvo na pilha caso a função chamar outra
função, uma vez que este mantem o endereço de retorno da função
chamadora.
\item Os registradores \textit{a0 - a3} não são salvos na função chamada.
Se for necessário a sua preservação, é papel da função chamadora fazer
isso antes de realizar uma chamada.
\item Os registradores \textit{v0 - v7}, \textit{fp}, \textit{sp} e
\textit{lr} devem ser preservados na função chamada, caso forem modificados.
Uma função que modifica tais registradores deve salvar seus valores na
no início da função e restaurar os mesmos no final, antes do retorno.
Apenas registradores modificados precisam ser preservados.
\item Funções simples que requerem apenas 4 registradores de trabalho
(\textit{a0 - a3}) e que não chamam outras funções não precisam salvar
registradores na pilha, desde que os registradores preservados não forem
modificados.
\end{itemize}

\subsubsection{Procedimento para chamadas de função}

\begin{itemize}
\item Na função chamadora, são realizados os seguintes passos: a) Passar os
argumentos nos registradores \textit{a0 - a3}; b) Configurar \textit{lr}
com o endereço de retorno; c) Saltar para a função; d) No retorno, ler
os valores retornados de \textit{a0 - a3}.
\item Na função chamada, são realizados os seguintes passos: a) Salvar os
registradores \textit{v0 - v7}, \textit{fp} e \textit{lr} na pilha,
conforme necessário; b) (opcional) Copiar \textit{sp} para \textit{fp};
c) (opcional) Alocar uma região da pilha para variáveis locais
(acessíveis via \textit{fp}); d) Realizar o processamento; e) Passar os
valores de retorno em \textit{a0 - a3}; f) (opcional) Desalocar a pilha;
g) Restaurar os registradores salvos (\textit{lr}, \textit{fp},
\textit{v7 - v0}) da pilha; h) Retornar para a função chamadora usando
\textit{lr}.
\end{itemize}

O exemplo abaixo ilustra o procedimento. Na função chamada não é feito
o salvamento dos registradores \textit{v0 - v7}, tampouco \textit{fp}
ou \textit{lr} pois os mesmos não são modificados. Os argumentos são
passados por valor.

\lstinputlisting[caption=fcall.s]{src/fcall.s}

No exemplo a seguir, os parâmetros são passados por referência. Dentro
da função chamada, os valores são acessados por ponteiros.

\lstinputlisting[caption=fcallbyref.s]{src/fcallbyref.s}

É possível realizar a passagem de mais do que quatro parâmetros para
uma função. Para isso, deve-se colocar os argumentos na pilha na função
chamadora em sua ordem inversa, e passar uma referência para o endereço
do primeiro parâmetro como argumento. Os argumentos podem ser acessados
na função chamada por valor ou referência. Cabe à função chamadora
gerenciar a pilha, desalocando os valores previamente alocados.

No exemplo abaixo, a função principal aloca o espaço de 6 bytes na
pilha, para passagem de 3 parâmetros. Os parâmetros são passados em
ordem inversa, e o ponteiro de pilha é copiado para o registrador
\textit{a0}. Na função chamada, os parâmetros são acessados na pilha
por meio de uma referência.

\lstinputlisting[caption=fcallstack.s]{src/fcallstack.s}

\subsection{Funções aritméticas}

A arquitetura TRM não possui suporte nativo para operações de multiplicação,
divisão ou resto da divisão (módulo). Essas operações precisam ser
sintetizadas por funções de biblioteca. Abaixo são apresentadas
implementações eficientes para essas operações\footnote{A implementação
das operações de divisão e módulo ocorre na mesma função, ou seja, a
divisão já retorna o quociente e o módulo.}. É importante observar que
as implementações poderiam ser otimizadas, substituindo-se referências
a constantes (o número 1, por exemplo) por um registrador com esse valor,
entre outras melhorias. Estas são apresentadas dessa forma por motivos
de simplicidade e melhor legibilidade do código.

\subsubsection{Multiplicação}

Para a multiplicação, devem ser passados como parâmetros dois números
(multiplicando e multiplicador) nos registradores \textit{a0} e \textit{a1}
respectivamente, sendo o produto resultante armazenado no registrador
\textit{a0}. Os operandos podem ser ou não sinalizados.

\lstinputlisting[caption=mul.s]{src/mul.s}

\subsubsection{Divisão e módulo}

Para a divisão e módulo, devem ser passados como parâmetros dois números
(numerador e denominador) nos registradores \textit{a0} e \textit{a1}
respectivamente, sendo o quociente resultante armazenado no registrador
\textit{a0} e o módulo no registrador \textit{a1}. Os operandos podem
ser ou não sinalizados, devendo ser utilizada a versão correta da
função de acordo.

\lstinputlisting[caption=div.s]{src/div.s}

\chapter{Ferramenta de montagem e simulação}

\section{Exemplos}

Diversos exemplos para uso com a ferramenta de montagem e simulação
estão disponíveis no diretório \textit{trm/examples/}. Todos são
programas descritos com a sintaxe da linguagem de montagem apresentada
no capítulo anterior.

\section{Compilação da ferramenta}

O código fonte da ferramenta de montagem e simulação está disponível no
diretório \textit{trm/java/}. Para compilar a ferramenta, é necessário
ter o pacote \textit{jdk-11} (ou posterior) instalado em seu sistema,
além de ferramentas básicas como \textit{make}. O processo é simples de
ser realizado a partir da linha de comando:

\begin{verbatim}
$ cd trm/java
$ make
$ cd ..
\end{verbatim}

Existem duas versões da ferramenta. A primeira possui uma interface 
gráfica que integra um simples editor, montador e simulador da arquitetura.
A segunda possui uma interface que pode ser controlada por linha de
comando. Para carregar a ferramenta gráfica, utilize o comando:

\begin{verbatim}
$ java -jar java/TrmGUI.jar
\end{verbatim}

Irá ser apresentada a interface gráfica da ferramenta. As funcionalidades
de edição, montagem e simulação podem ser selecionada por diferentes
\textit{tabs}. A ferramenta pode de linha de comando pode ser invocada
com:

\begin{verbatim}
$ java -jar java/Trm.jar
\end{verbatim}

Será apresentado um conjunto de parâmetros a serem passados à ferramenta
para o uso de diferentes modos de operação da mesma. É possível realizar
apenas o processo de montagem e geração de um arquivo com o código objeto
(modo 'a'), simulação do código objeto gerado anteriormente, a partir
de um arquivo (modo 's'), os processos combinados de montagem e simulação,
sem a geração de um arquivo intermediário (modo 'r') e por último um
sistema de depuração e execução interativa (modo 'd').

\begin{verbatim}
Syntax: java -jar Trm.jar <mode> <source> <output>
<mode>   - a (assemble), s (simulation), r (run), d (debug)
<source> - assembly (.s) source code for 'a', 'r' and 'd' modes,
           object code (.trm) for 's' mode
<output> - object code (.trm) for 'a' mode

\end{verbatim}

\section{Montagem}

Na ferramenta gráfica, o processo de montagem é simples, bastando 
inicialmente carregar um código fonte no editor de texto (\textit{Code
Editor})\footnote{Esse processo pode ser realizado selecionando-se a
opção \textit{File - Open} ou colando-se um texto no editor diretamente.}
e selecionar a tab \textit{Assembler and Simulator}. Para montar o código,
seleciona-se a opção \textit{Assemble/Exec}, acessível por meio do menu
superior ou botão no lado direito da interface. Após a montagem do
programa, será apresentada a listagem do código objeto, tabela de 
símbolos e opções para simulação.

Para a ferramenta de linha de comando, o processo de montagem de um
programa pode ser realizado fornecendo-se à ferramenta o código fonte e
um arquivo de saída. No arquivo de saída será armazenado o código objeto
gerado e eventuais erros. Exemplo:

\begin{verbatim}
$ java -jar java/Trm.jar a examples/01-add.s 01-add.txt
\end{verbatim}

Caso não seja fornecido um arquivo de saída, a mesma será redirecionada
para a saída padrão. Para o código fonte do exemplo, será gerada essa
saída:

\begin{verbatim}
0000 0950    (add r5,r0,123)
0002 007b
0004 0955    (add r5,r5,456)
0006 01c8
0008 4950    (stw r5,r0,61440)
000a f000
000c 4160    (ldw r6,r0,28)
000e 001c
0010 4170    (ldw r7,r0,30)
0012 001e
0014 0867    (add r6,r7)
0016 4960    (stw r6,r0,61440)
0018 f000
001a f800    (hlt r0,r0)
001c 007b
001e 01c8
\end{verbatim}

\section{Simulação}

A partir do código objeto gerado no passo anterior, pode-se realizar uma
simulação do mesmo. O simulador irá verificar se o código objeto está
correto antes de iniciar a simulação.

Na ferramenta gráfica, a simulação pode ser realizada de uma só vez
(opção \textit{Run}), passo a passo de forma automática (opção
\textit{Auto Step}) ou manualmente (opção \textit{Stop/Step}). O processo
de simulação na ferramenta gráfica já inclui depuração. A depuração
inclui um acompanhamento da execução, onde é feito um apontamento da
instrução na listagem do código objeto antes de sua execução, bem como
referências a símbolos. Após a execução de uma instrução, é atualizado
o banco de registradores, o contador de programa, a memória (caso a
última instrução referencie a mesma) e é feito o apontamento da próxima
instrução a ser executada.

Na ferramenta por linha de comando, a simulação pode ser realizada
passando-se como argumento a opção de simulação e um arquivo com o código
objeto.

\begin{verbatim}
$ java -jar java/Trm.jar s 01-add.txt
[program (code + data): 32 bytes]
[memory size: 61440]
579
579
[halt]
8 cycles
\end{verbatim}

\section{Modo combinado (montagem e simulação, linha de comando)}

O modo de execução de programas combinado é composto pela montagem e
simulação em um único passo e existe na ferramenta de linha de comando.
Esse modo pode ser útil quando se deseja desenvolver uma aplicação, pois
evita-se um passo intermediário. Para invocar o modo combinado para
execução do mesmo exemplo anterior, pode ser utilizado o seguinte comando:

\begin{verbatim}
$ java -jar java/Trm.jar r examples/01-add.s
\end{verbatim}

O resultado do comando será o seguinte:

\begin{verbatim}
...
0014 0867    (add r6,r7)
0016 4960    (stw r6,r0,61440)
0018 f000
001a f800    (hlt r0,r0)
001c 007b
001e 01c8
[program (code + data): 32 bytes]
[memory size: 61440]
579
579
[halt]
8 cycles
\end{verbatim}

\section{Modo depuração (linha de comando)}

No modo depuração, assim como no modo combinado, são realizados os
processos de montagem e simulação do código em um único passo (ferramenta
de linha de comando). A diferença do modo depuração para o modo combinado
é que, o segundo executa o programa até o final já o primeiro, permite
que o usuário execute o programa de forma interativa, instrução por
instrução. Além disso, esse modo possui recursos como endereço para parada
(\textit{breakpoint}), reinicio do programa, listagem do programa,
listagem da tabela de símbolos e visualização da memória (\textit{memory
dump}). Para o programa usado nos exemplos anteriores, o processo é
iniciado com o comando:

\begin{verbatim}
$ java -jar java/Trm.jar d examples/01-add.s
\end{verbatim}

O resultado do comando será o seguinte:

\begin{verbatim}
[program (code + data): 32 bytes]
[memory size: 61440]
pc: 0000, instruction: add r5,r0,123
 r0: [0000]  r1: [0000]  r2: [0000]  r3: [0000]
 r4: [0000]  r5: [0000]  r6: [0000]  r7: [0000]
 r8: [0000]  r9: [0000] r10: [0000] r11: [0000]
r12: [0000] r13: [0000] r14: [f000] r15: [0000]
q: quit; r: run; b: breakpoint; t: reset program; ENTER: next;
l: program list; s: symbol table; d: mem dump
\end{verbatim}

\section{Serviços do simulador}

Como parte das funcionalidades do simulador, está implementado um pequeno
conjunto de serviços, que pode ser utilizado para entrada e saída de
dados em diferentes formatos. Os serviços de saída são acessados por
meio de uma operação de escrita na memória (\textit{stw}) e os serviços
de entrada por meio de uma leitura (\textit{ldw}). Os serviços podem ser
utilizados por meio de instruções no formato regular ou imediato.
Exemplos:

\begin{verbatim}
    add v0,zr,1234              ; v0 = 1234
    add v1,zr,65                ; v0 = 'A'
    stw v0,zr,0xf000            ; write int (with line feed)
    stw v1,zr,0xf006            ; write char
    stw v0,zr,0xf00c            ; write hex (with line feed)
    ...
    ldw v0,zr,0xf010            ; read int
    add v0,zr,buf               ; v0 = &buf
    ldw v0,zr,0xf018            ; read string
    add v0,zr,buf               ; v0 = &buf
    stw v0,zr,0xf008            ; write string
\end{verbatim}

Em todas as operações de saída, o primeiro registrador da instrução deve
conter o valor a ser apresentado na saída, salvo operações de impressão
de \textit{strings}, que devem conter o endereço de memória onde está
a mesma. Em operações de leitura, o primeiro registrador da instrução
conterá o valor lido da entrada de dados do usuário. Leituras de
\textit{strings} devem ter armazenado nesse registrador o endereço de
memória onde será armazenada a entrada de dados do usuário. A tabela
abaixo apresenta os endereços de cada serviço, bem como a operação que
deve ser realizada (escrita (W), instrução \textit{stw} ou leitura (R),
instrução \textit{ldw}). A tabela abaixo apresenta os serviços definidos
no simulador.

\begin{table}[htb!]
\centering
\begin{tabular}{|p{2.0cm}|p{9.0cm}|p{2.0cm}|}
\hline
\bf{Endereço} & \bf{Serviço} & \bf{Operação} \\ \hline \hline
\texttt{0xf000} & \texttt{write int (with line feed)} & \texttt{W} \\ \hline
\texttt{0xf002} & \texttt{write int} & \texttt{W} \\ \hline
\texttt{0xf004} & \texttt{write char (with line feed)} & \texttt{W} \\ \hline
\texttt{0xf006} & \texttt{write char} & \texttt{W} \\ \hline
\texttt{0xf008} & \texttt{write string (with line feed)} & \texttt{W} \\ \hline
\texttt{0xf00a} & \texttt{write string} & \texttt{W} \\ \hline
\texttt{0xf00c} & \texttt{write hex (with line feed)} & \texttt{W} \\ \hline
\texttt{0xf00e} & \texttt{write hex} & \texttt{W} \\ \hline
\texttt{0xf010} & \texttt{read int} & \texttt{R} \\ \hline
\texttt{0xf014} & \texttt{read char} & \texttt{R} \\ \hline
\texttt{0xf018} & \texttt{read string} & \texttt{R} \\ \hline
\texttt{0xf01c} & \texttt{read hex} & \texttt{R} \\ \hline
\end{tabular}
\end{table}

\end{document}
